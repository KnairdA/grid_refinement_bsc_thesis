\newtheorem{Satz}{Satz}[section]
\newtheorem{Lemma}[Satz]{Lemma}

\theoremstyle{definition}
\newtheorem{Definition}{Definition}[section]

\numberwithin{equation}{section}

\newcommand{\R}{\mathbb{R}} % reelle
\newcommand{\Z}{\mathbb{Z}} % ganze
\newcommand{\N}{\mathbb{N}} % natuerliche

\newcommand{\G}{\mathcal{G}} % Grobes Gitter
\newcommand{\F}{\mathcal{F}} % Feines Gitter
\newcommand{\U}{\mathcal{U}} % Übergangsbereich

\newcommand{\restrictedto}[2]{{\left.\kern-\nulldelimiterspace #1 \vphantom{\big|} \right|_{#2}}}

\DeclareMathOperator{\res}{\boldsymbol{r}}
\newcommand{\resarg}[2]{\boldsymbol{r(}{#1}, {#2}\boldsymbol{)}}

\DeclareMathOperator{\ipol}{\boldsymbol{n}}
\DeclareMathOperator{\sipol}{\overline{\boldsymbol{n}}}
\newcommand{\ipolarg}[2]{\boldsymbol{n}_{#1}\boldsymbol{(}{#2}\boldsymbol{)}}
\newcommand{\sipolarg}[1]{\overline{\boldsymbol{n}}\boldsymbol{(}{#1}\boldsymbol{)}}
\newcommand{\sipolderivarg}[2]{\overline{\boldsymbol{n}}^{\,(#1)}\boldsymbol{(}{#2}\boldsymbol{)}}

\newcommand{\class}[1]{\texttt{#1}}
\newcommand{\method}[1]{\texttt{#1}}

\newcommand\numberthis{\addtocounter{equation}{1}\tag{\theequation}}

\newcommand{\V}[2]{\ensuremath{\begin{pmatrix}#1\\#2\end{pmatrix}}}
\newenvironment{rcases}{\left.\begin{aligned}}{\end{aligned}\right\rbrace}

% See: https://tex.stackexchange.com/questions/108140/draw-3d-rectangle
\makeatletter
\tikzoption{canvas is xy plane at z}[]{%
	\def\tikz@plane@origin{\pgfpointxyz{0}{0}{#1}}%
	\def\tikz@plane@x{\pgfpointxyz{1}{0}{#1}}%
	\def\tikz@plane@y{\pgfpointxyz{0}{1}{#1}}%
	\tikz@canvas@is@plane
}
\makeatother

\tikzset{xyp/.style={canvas is xy plane at z=#1}}
\tikzset{xzp/.style={canvas is xz plane at y=#1}}
\tikzset{yzp/.style={canvas is yz plane at x=#1}}
