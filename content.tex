\section{Einführung}

\subsection{Wieso Strömungen simulieren?}

Das hoch technisierte Lebensumfeld des modernen Menschen ist ohne ein detailliertes Verständnis des Verhaltens der, durch ihn in wachsendem Maße kontrollierten, Natur undenkbar. Eine wichtige Komponente dieses Naturverständnisses ist das Wissen um das Verhalten von Flüssigkeiten und Gasen, die sich als Strömungen bewegen. Ohne dieses Verständnis führe kein Automobil, flöge kein Flugzeug und drehten sich weder ein Windrad um seine Achse noch ein Satellit um unseren Planeten.

Experimente zur Bestimmung des Verhaltens von Strömungen -- z.B. in Wind- und Wasserkanälen -- sind möglich, liefern naturnaheste Ergebnisse und stellen eine zentrale Komponente der Entwicklung eben genannter Errungenschaften dar. Leider sind reale Experimente im Allgemeinen nicht nur sehr aufwändig in Aufbau und Durchführung, sondern stoßen auch insbesondere bei der Betrachtung mikroskopischer Probleme -- etwa im Bereich der Medizin, deren menschliches Subjekt zu einem großen Teil ebenfalls eine \emph{Strömungsmaschine} bildet -- an Grenzen von Messmethoden und Ethik.

So trifft es sich, dass der Wunsch nach theoretischer Lösung komplexer und nur schwer analytisch zugänglicher Strömungsprobleme mit der Entwicklung von immer leistungsfähigeren Rechenmaschinen nicht nur einherging sondern auch eine der Triebfedern in deren initialen Entwicklung war. Moderne numerische Verfahren zur Simulation von Strömungen versprechen eine zunehmende Reduzierung benötigter realer Experimente und sind heute gängiges Werkzeug in Forschung und Maschinenbau.

\subsection{Weshalb mit Lattice Boltzmann Methoden?}

Während Finite Elemente Methoden den wohl verbreitetsten Ansatz zur numerischen Strömungsdynamik bilden, erfreut sich auch die Herangehensweise der Lattice Boltzmann Methoden in den letzten Jahrzehnten wachsender Nutzbarkeit und Verbreitung. Im Gegensatz zu anderen Lösungsmethoden werden hier die strömungsbeschreibenden Navier-Stokes Gleichungen nicht direkt numerisch gelöst. Lösungen ergeben sich vielmehr aus der Simulation des Fluidverhaltens auf \emph{mesoskopischer} Ebene -- d.h. aus der Betrachtung nicht aus Sicht der Kollision einzelner Fluidpartikel und nicht aus Sicht der analytischen Strömungsbeschreibung, sondern aus Sicht der Wahrscheinlichkeit, dass sich das Fluid zu bestimmter Zeit an einem bestimmten Ort mit bestimmter Geschwindigkeit bewegt.

\bigskip
Ein Vorteil dieses, auf den Arbeiten von Ludwig Eduard Boltzmann im Bereich der statistischen Physik aufbauenden, Ansatzes ist seine Eignung für komplexe Geometrien mit verschiedensten Randbedingungen. Weiterhin gewinnt in den letzten Jahren auch die sehr gute Parallelisierbarkeit von Lattice Boltzmann Methoden in Hinblick auf einen technischen Fortschritt an Anziehungskraft, nach welchem die Leistungsfähigkeit von Großrechnern eher aus deren Parallelität als aus individueller Prozessorleistung erwächst.

\subsection{Warum Gitterverfeinerung?}

Die einfachsten und zugleich am weitesten verbreiteten Umsetzungen von Simulationen mit Lattice Boltzmann Methoden basieren auf uniformen Gittern, in denen Zellen immer den gleichen Abstand zu ihren Nachbarzellen haben.

Die Genauigkeit von Lattice Boltzmann basierenden Simulationen hängt maßgeblich von der Auflösung des verwendeten Gitters ab. Bei Außerachtlassung weiterer wichtiger Faktoren wie dem verwendeten Kollisionsterm und Randkonditionen kann im Allgemeinen davon ausgegangen werden, dass eine feinere Auflösung des Gitters zu besseren Ergebnissen führt.

In praktischen Beispielen können innerhalb eines Modells große Unterschiede in der Strömungskomplexität existieren. So kann es große Gebiete eines Modells geben, die mit einem vergleichsweise groben Gitter gut simuliert werden können, während in anderen Gebieten -- beispielsweise in komplexen Geometrien und an Rändern -- ein vielfach feineres Gitter zur adäquaten Behandlung benötigt wird. In uniformen Gittern muss jedoch das gesamte Modell unabhängig der lokalen Situation mit der maximal benötigten Auflösung abgebildet werden.

Da die Anzahl der benötigten Gitterpunkte sich maßgeblich auf den Speicherbedarf und Rechenaufwand auswirkt, ist es wünschenswert die Anzahl der Gitterpunkte zu minimieren. Ein Ansatz, dies zu erreichen, ist die lokale Variation der Gitterauflösung.

\newpage
\section{Grundlagen}

In diesem Kapitel werden wir die, dem weiteren Verlauf dieser Arbeit zugrunde liegende, Lattice Boltzmann Methode in 2D nachvollziehen.

\subsection{Lattice Boltzmann Methode}\label{kap:LBM}

Grundlage und Namensgeber von Simulationen mit Lattice Boltzmann Methoden ist die Boltzmann Gleichung. Sie beschreibt das Verhalten von Gasen auf mesoskopischer Ebene als Verteilungsfunktion der Masse von Partikeln in einer Raumregion mit gegebener Geschwindigkeit.

\begin{Definition}[Die Boltzmann-Gleichung]
Sei \(f(x,\xi,t)\) die Verteilungsfunktion der Partikelmasse zu Zeit \(t\) in Ort \(x \in \R^2\) mit Geschwindigkeit \(\xi \in \R^2\), \(\rho\) die Dichte und \(F \in \R^2\) eine etwaige äußere Kraft. Die Boltzmann-Gleichung beschreibt die zeitliche Veränderung der Verteilungsfunktion anhand des totalen Differential \(\Omega(f)\):
\[ \left( \partial_t + \xi \cdot \partial_x + \frac{F}{\rho} \cdot \partial_\xi \right) f = \Omega(f) \left( = \partial_x f \cdot \frac{dx}{dt} + \partial_\xi f \cdot \frac{d\xi}{dt} + \partial_t f \right) \]

Hierbei handelt es sich um eine Advektionsgleichung wobei der Term \(\partial_t f + \xi \cdot \partial_x f\) die Strömung der Partikelverteilung mit Geschwindigkeit \(\xi\) und \(\frac{F}{\rho} \cdot \partial_\xi f\) einwirkende Kräfte darstellt. Der Term \(\Omega(f)\) beschreibt, entsprechend als Kollisionsoperator bezeichnet, die kollisionsbedingte lokale Neuverteilung von \(f\).
\end{Definition}

Zentrale Anforderung an den Kollisionsoperator ist die Impuls- und Masseerhaltung. Die im Folgenden betrachtete Lattice Boltzmann Methode verwendet die übliche BGK Approximation der Boltzmann-Gleichung ohne äußere Kraft von Bhatnagar, Gross und Krook (siehe \citetitle{krueger17}~\cite[Kap.~3.5.3]{krueger17}).
Grundlegendes Element dieser Approximation ist der BGK Operator
\[\Omega(f) := -\frac{f-f^\text{eq}}{\tau} \Delta t\]
welcher die Partikelverteilung mit Rate \(\tau\) gegen eine Equilibriumsverteilung \(f^\text{eq}\) relaxiert. Ohne Beschränkung der Allgemeinheit setzen wir dabei  im Folgenden \(\Delta t = 1\).

Wenden wir den BGK Operator auf die Boltzmann-Gleichung ohne äußere Kräfte an, erhalten wir die BGK Approximation:

\begin{Definition}[BGK Approximation]
Sei \(\tau \in \R_{\geq 0}\) eine Relaxionszeit und \(f^\text{eq}\) die von der Maxwell-Boltzmann-Verteilung gegebene Equilibriumsverteilung.
\[ (\partial_t + \xi \cdot \nabla_x) f = -\frac{1}{\tau} (f(x,\xi,t) - f^\text{eq}(x,\xi,t)) \]
\end{Definition}

Analog zur Boltzmann-Gleichung ist auch bei deren BGK Approximation der beschriebene Ort \(x \in \R^2\) im Allgemeinen frei gewählt. Da unser Ziel jedoch gerade die Diskretisierung der Simulationsdomäne in einem Gitter ist, wollen wir \(x\) einschränken:

\begin{Definition}[Ortsdiskretisierung]
\label{def:SpatialDiscretizationLBM}
Sei die örtliche Simulationsdomäne \(D := \R^2\) diskretisiert als kartesisches Gitter mit Zellabstand \(\Delta x \in \R_+\). Dann ist die Einbettung der kartesischen Gitterdomäne \(L := \Z^2\) in \(D\) gegeben als:
\[d : L \to D, \ l \mapsto \Delta x \cdot l\]
Analog zur o.B.d.A. erfolgten Wahl von \(\Delta t = 1\) setzen wir auch hier \(\Delta x = 1\), sodass wir die Simulations- und Gitterdomäne -- bei Inklusion stetiger Übergänge zwischen Elementen aus \(L\) -- identisch identifizieren können. Der stetige Übergang zwischen zwei orthodonal benachbarten Gitterknoten erfolgt somit auf dem Einheitsintervall. Praktisch können wir also im Folgenden bei Annahme von \(x \in L \subset D\) die explizite Ausführung der Gittereinbettung vernachlässigen.
\end{Definition}

Zu erwähnen ist an dieser Stelle die Wichtigkeit einer klaren Unterscheidung zwischen Simulationsdomäne und dem durch diese zu modellierenden physikalischen System für die konkrete Interpretation des Simulationsergebnisses \cite[Kap.~7]{krueger17}.

Für die verbleibende Herleitung der LBM können wir diese Interpretation, d.h. die Unterscheidung zwischen physikalischen- und Lattice-Einheiten, jedoch außer Acht lassen, da sich die modellierten physikalischen Einheiten aus der Wahl der Relaxionszeit und der Skalierung der Lattice-Momente ergeben. Diese Wahl wird hier nicht weiter eingeschränkt.

\bigskip

Wir bemerken nun, dass die BGK Approximation nicht nur für beliebige Orte, sondern auch für beliebige Geschwindigkeiten \(\xi \in \R^2\) definiert ist. Da wir die LBM auf einem endlichen Rechner umsetzen wollen, müssen wir auch die Menge der betrachteten Geschwindigkeiten auf eine endliche Menge diskretisieren.

Eine übliche Menge diskreter Geschwindigkeiten in 2D ist \emph{D2Q9} wobei \emph{D2} die Anzahl der Dimensionen und \emph{Q9} die Anzahl der Geschwindigkeiten verschlüsselt.

\begin{Definition}[D2Q9 Modell]
\[ \{\xi_i\}_{i=0}^8 = \left\{ \V{0}{0}, \V{-1}{\phantom{-}1}, \V{-1}{\phantom{-}0}, \V{-1}{-1}, \V{\phantom{-}0}{-1}, \V{\phantom{-}1}{-1}, \V{1}{0}, \V{1}{1}, \V{0}{1} \right\} \]
\end{Definition}

\begin{figure}
\centering
\begin{tikzpicture}[
	scale=1.5,
	dot/.style={circle,draw=black,inner sep=3pt},
	center/.style={circle,fill=black,inner sep=4pt},
	arrow/.style={thick,->,>=stealth}
]

\node[dot,label=below left:\(\xi_3\)] at (0,0){ };
\node[dot,label=below:\(\xi_4\)] at (1,0){ };
\node[dot,label=below right:\(\xi_5\)] at (2,0){ };

\node[dot,label=left:\(\xi_2\)] at (0,1){ };
\node[center] at (1,1){ };
\node[dot,label=right:\(\xi_6\)] at (2,1){ };

\node[dot,label=above left:\(\xi_1\)] at (0,2){ };
\node[dot,label=above:\(\xi_8\)] at (1,2){ };
\node[dot,label=above right:\(\xi_7\)] at (2,2){ };

\draw[arrow] (1.0,1.2) -- (1.0,1.8);
\draw[arrow] (1.2,1.0) -- (1.8,1.0);
\draw[arrow] (1.0,0.8) -- (1.0,0.2);
\draw[arrow] (0.8,1.0) -- (0.2,1.0);

\draw[arrow] (1.2,1.2) -- (1.8,1.8);
\draw[arrow] (1.2,0.8) -- (1.8,0.2);
\draw[arrow] (0.8,0.8) -- (0.2,0.2);
\draw[arrow] (0.8,1.2) -- (0.2,1.8);
\end{tikzpicture}

\caption{Umgebung einer Zelle in D2Q9}
\end{figure}

Mithilfe einer solchen endlichen Menge diskreter Geschwindigkeiten lässt sich die BGK Approximation bezüglich der Geschwindigkeit diskretisieren:

\begin{Definition}[BGK Geschwindigkeitsdiskretisierung]
\label{def:disVelBGK}
Seien \(\xi_i\) Vektoren einer Menge mikroskopischer Geschwindigkeiten wie z.B. D2Q9 und \(f_i(x,t) \equiv f(x,\xi_i,t)\). Dann ist
\[ (\partial_t + \xi_i \cdot \nabla_x) f_i(x,t) = -\frac{1}{\tau} (f_i(x,t) - f_i^\text{eq}(x,t)) \]
die Diskretisierung der BGK Approximation entlang der Geschwindigkeiten in diskreten Gitterknoten \(x \in L\). Die Geschwindigkeiten müssen hier dank der Wahl von \(\Delta x = 1\) nicht weiter skaliert werden.
\end{Definition}

Hierbei ist die diskrete Equilibriumsverteilung \(f_i^\text{eq}\) wie folgt definiert:

\begin{Definition}[Diskrete Equilibriumsverteilung]
\label{def:fieq}
Seien \(\rho \in \R_{\geq 0}\) die Dichte, \(u \in \R^2\) die Gesamtgeschwindigkeit, \(\xi_i\) die \(i\)-te diskrete Geschwindigkeitskomponente, \(w_i\) das Gewicht jener Komponente bzgl. des Lattice und \(c_s\) die Lattice-Schallgeschwindigkeit.
\[f_i^\text{eq} = w_i \rho \left( 1 + \frac{u \cdot \xi_i}{c_s^2} + \frac{(u \cdot \xi_i)^2}{2c_s^4} - \frac{u \cdot u}{2c_s^2} \right)\]
\end{Definition}

Die Werte von \(u = u(x,t)\) und \(\rho = \rho(x,t)\) in Ort \(x\) zu Zeit \(t\) ergeben sich dabei aus den \emph{Momenten} der Verteilungsfunktion \(f_i\):

\begin{Definition}[Momente der Verteilungsfunktion]
\label{def:Momente}
\begin{align*}
\rho(x,t) &= \sum_{i=0}^{q-1} f_i(x,t) \\
\rho u(x,t) &= \sum_{i=0}^{q-1} \xi_i f_i(x,t)
\end{align*}
\end{Definition}

Für D2Q9 ergeben sich nach \cite[Gl.~3.60 bzw. Tab.~3.3]{krueger17} die Gewichte:
\[w_0 = \frac{4}{9}, \ w_{2,4,6,8} = \frac{1}{9}, \ w_{1,3,5,7} = \frac{1}{36}\]

Weiter folgt zusammen mit der Bedingung \(\sum_{i=1}^{q-1} w_i (\xi_i)_a (\xi_i)_b = c_s^2 \delta_{a,b}\) aus \cite[Gl.~3.60]{krueger17} die Schallgeschwindigkeit \(c_s = \sqrt{1/3}\) des Lattice. Konditionen zur Bestimmung dieser gitterspezifischen Konstanten sind hierbei die Erhaltung von Impuls und Masse sowie die Forderung von \emph{Rotationsisotropie}.

Zur Entwicklung einer \emph{implementierbaren} expliziten BGK Gleichung können wir nun die Geschwindigkeitsdiskretisierung~\ref{def:disVelBGK} integrieren:
\[ f_i(x+\xi_i, t+1) - f_i(x,t) = \int_0^1 \Omega_i(x+\xi_i s,t+s) ds \]

Wobei \(\Omega_i(x,t)\) hier die diskrete Formulierung des BGK Kollisionsoperators darstellt:
\[\Omega_i(x,t) := -\frac{1}{\tau} ( f_i(x,t) - f_i^\text{eq}(x, t) )\]

Da sich die exakte Lösung des Integrals auf der rechten Seite schwierig gestaltet, wird es in der Praxis nur approximiert. Während es dazu vielfältige Ansätze gibt, beschränken wir uns an dieser Stelle auf Anwendung der Trapezregel:
\begin{align*}
f_i(x+\xi_i,t+1) - f_i(x,t) &= \frac{1}{2} \left( \Omega_i(x,t) + \Omega_i(x+\xi_i,t+1) \right) \\
&= -\frac{1}{2\tau} \left( f_i(x+\xi_i,t+1) + f_i(x,t) - f_i^\text{eq}(x+\xi_i,t+1) - f_i^\text{eq}(x,t) \right)
\end{align*}

Zur expliziten Lösung dieser impliziten Gleichung benötigen wir nun nur noch eine geeignete Verschiebung von \(f_i\) und \(\tau\):

\begin{Definition}[Diskrete LBM BGK Gleichung]
\label{def:LBGKeq}
Seien \(\overline{f_i}\) und \(\overline\tau\) definiert:
\begin{align*}
\overline{f_i} &= f_i + \frac{1}{2\tau}(f_i - f_i^\text{eq}) \\
\overline\tau &= \tau + \frac{1}{2}
\end{align*}

Setzen wir diese verschobenen Variablen in das Ergebnis der Trapezregel ein, erhalten \cite[Kap.~A.5 mit \(\Delta t=1\)]{krueger17} wir die die vollständig diskretisierte LBM BGK Gleichung:

\[\overline{f_i}(x+\xi_i,t+1) = \overline{f_i}(x,t) - \frac{1}{\overline\tau} (\overline{f_i}(x,t) - f_i^\text{eq}(x,t))\]
\end{Definition}

Bemerkenswert ist an dieser Stelle, dass die Momente der Verteilungen mit \(\overline{f_i}\) analog zu \ref{def:Momente} berechnet werden können:
\begin{align*}
\sum_{i=0}^{q-1} \overline{f_i} = \sum_{i=0}^{q-1} f_i + \frac{1}{2\tau} \sum_{i=0}^{q-1} (f_i - f_i^\text{eq}) &= \rho \\
\sum_{i=0}^{q-1} \xi_i \overline{f_i} = \sum_{i=0}^{q-1} \xi_i f_i + \frac{1}{2\tau} \sum_{i=0}^{q-1} (f_i - f_i^\text{eq}) &= \rho u
\end{align*}

\newpage
\subsubsection{Algorithmus}\label{kap:LBMimpl}

Bei der Implementierung der {diskreten LBM BGK Gleichung}~\ref{def:LBGKeq} auf einem Computer ist die Aufteilung in Kollisions- und Strömungsschritt üblich.

\begin{Definition}[Kollisionsschritt]
Annäherung der Verteilungsfunktion an die lokal berechnete Equilibriumsverteilung entsprechend dem BGK Kollisionsoperator.
\[f_i^\text{out}(x,t) = f_i(x,t) - \frac{1}{\tau}(f_i(x,t) - f_i^\text{eq}(x,t))\]
\end{Definition}

\begin{Definition}[Strömungsschritt]
Strömen der neuen Verteilungen auf die benachbarten Zellen entsprechend der jeweiligen diskreten Geschwindigkeit.
\[f_i(x+\xi_i,t+1) = f_i^\text{out}(x,t)\]
\end{Definition}

Bemerkenswert ist hierbei, dass der Kollisionsschritt nur lokale Informationen der jeweiligen Zelle benötigt und sich somit sehr gut zur parallelen Verarbeitung eignet.

\subsubsection{Chapman-Enskog Analyse}

Ziel der beschriebenen Lattice Boltzmann Methode ist die möglichst gute Approximation der schwach-kompressiblen Navier-Stokes Gleichungen auf der Simulationsdomäne.

\begin{Definition}[Schwach-kompressible Navier-Stokes Gleichungen]
Sei \(\rho\) die Dichte, \(u\) die Geschwindigkeit und \(p\) der Druck zu Zeit \(t\) sowie \(\nu\) die kinematische Viskosität und \(\mathrm{S}\) der Verzerrungstensor.
\begin{align*}
\partial_t  \rho + \nabla \cdot (\rho u) &= 0 \\
\partial_t u + (u \cdot \nabla) u &= -\frac{1}{\rho} \nabla p + 2\nu\nabla \cdot (\mathrm{S})
\end{align*}

Dabei sind Druck \(p\), kinetische Viskosität \(\nu\) und Verzerrungstensor \(\mathrm{S}\) definiert als:
\begin{align*}
p &= c_s^2 \rho \\
\nu &= c_s^2 \tau \\
\mathrm{S} &= \frac{1}{2} (\nabla u + (\nabla u)^\top) \numberthis\label{eq:Verzerrungstensor}
\end{align*}
\end{Definition}

Nach \cite[Kap.~4.1]{krueger17} kann die asymptotische Äquivalenz von LBM BGK Gleichung und schwach-kompressiblen Navier-Stokes Gleichungen mit der Entwicklung von Chapman-Enskog gezeigt werden.

\begin{Definition}[Chapman-Enskog Ansatz]
\label{def:ChapmanEnskog}
Der Chapman-Enskog Ansatz besteht in der Annahme, dass die Verteilungsfunktion \(f_i\) als leicht gestörte Equilibriumsverteilung dargestellt werden kann: \[f_i = f_i^\text{eq} + \epsilon f_i^{(1)} + \mathcal{O}(\epsilon^2)\]
Hierbei ist \(\epsilon f_i^{(1)}\) mit \(\epsilon \ll 1\) der Störterm 1. Ordnung. Dieser ist gegeben als:
\[\epsilon f_i^{(1)} = \frac{w_i}{2 c_s^4} \mathrm{Q}_i : \mathrm{\Pi}^{(1)} \numberthis\label{eq:firstOrderPertubation}\]
Wobei der Geschwindigkeitstensor \(\mathrm{Q}_i\) und das Störmoment \(\mathrm\Pi^{(1)}\) nach \cite[Kap.~4.1.3]{krueger17} definiert sind als:
\begin{align*}
\mathrm{Q}_i &= \xi_i \xi_i - c_s^2 I \numberthis\label{eq:velocityTensor} \\
\mathrm\Pi^{(1)} &= \sum_{i=0}^{q-1} \xi_i \xi_i \epsilon f_i^{(1)} = -2 c_s^2 \rho \overline\tau \mathrm{S} \numberthis\label{eq:pertubationMoment}
\end{align*}
\end{Definition}

\begin{Definition}[Nicht-Equilibriumsverteilung]
Die Relaxion der Verteilungsfunktion \(f_i\) gegen die Equilibriumsverteilung \(f_i^\text{eq}\) impliziert eine Dekomposition in Equilibriums- und Nicht-Equilibriumsverteilung: \[f_i := f_i^\text{eq} + f_i^\text{neq}\]
\end{Definition}

Unter Vernachlässigung von Störtermen der Ordnung \(\mathcal{O}(\epsilon^2)\) ergibt sich eine Näherung der Nicht-Equilibriumsverteilung:
\[f_i^\text{neq} \cong \epsilon f_i^{(1)}\]

Diese Darstellung können wir unter Verwendung von Definition~\ref{def:ChapmanEnskog} ausführen als:
\[f_i^\text{neq} \cong -\frac{w_i \rho \overline\tau}{c_s^2} \mathrm{Q}_i : \mathrm{S} \numberthis\label{eq:approxFneq}\]

\newpage
\subsection{Herangehensweisen an Gitterverfeinerung}

Grundsätzlich existieren mit der Multi-Grid bzw. Multi-Domain Herangehenweise zwei verschiedene Ansätze für Gitterverfeinerung in Lattice Boltzmann Methoden \cite[Kap.~3.1]{lagrava12}. Im Westenlichen unterscheiden die Ansätze sich in der Struktur der variabel aufgelösten Teilgitter der Simulationsdomäne. Weitere Unterschiede folgen dann aus dieser grundlegenden Struktur.

\subsubsection{Multi-Grid Ansatz}

Bei Verfahren des Multi-Grid Ansatzes existiert das gröbste Gitter über der gesamten Domäne. Feinere Teilgitter werden über gröberen Gittern platziert und nicht aus deren Verarbeitung ausgeschlossen. Somit existieren über der gesamten \emph{Fläche} feinerer Gitter auch die Knoten gröberer Gitter.

\begin{figure}[h]
\centering
\begin{tikzpicture}[
	scale=0.4,
	coarse/.style={circle,draw=black,inner sep=2},
	fine/.style={cross out,draw=black,inner sep=1},
]

\foreach \x in {0,...,7}
	\foreach \y in {0,...,7}
		\node[coarse] at (\x,\y){ };

\draw (9,3.5) node[cross out,draw=black,line width=0.5mm,rotate=45]{};

\foreach \x in {0,...,8}
	\foreach \y in {0,...,6}
		\node[fine] at (11+0.5*\x,2+0.5*\y){ };

\draw[->,line width=0.5mm] (17.5,3.5) -- (19.5,3.5);

\foreach \x in {22,...,29}
	\foreach \y in {0,...,7}
		\node[coarse] at (\x,\y){ };

\foreach \x in {0,...,8}
	\foreach \y in {0,...,6}
		\node[fine] at (25+0.5*\x,2+0.5*\y){ };
\end{tikzpicture}

\caption{Teilgitter in der Multi-Grid Herangehensweise}
\end{figure}

Vorteil dieser Herangehensweise ist es, dass feinere Teilgitter im Verlauf der Simulation ohne aufwendige Restrukturierung verschoben werden können, um beispielsweise komplexere Strömungsstrukturen zu \emph{verfolgen}. Nachteil ist, dass das Einsparpotentiale in Speicher- und Rechenbedarf durch Mehrfachabdeckung von Teilen der Simulationsdomäne mittels mehrerer Gitter nicht voll ausgenutzt werden können.

\subsubsection{Multi-Domain Ansatz}

Kern des Multi-Domain Ansatzes ist es, außerhalb von etwaigen verfahrensbedingten Übergangsbereichen, jede Position in der Simulationsdomäne durch genau ein Teilgitter abzubilden. Konkret werden also bereits durch feinere Gitter abgedeckte Bereiche aus gröberen Teilgittern ausgespart.

\begin{figure}[h]
\centering
\begin{tikzpicture}[
	scale=0.4,
	coarse/.style={circle,draw=black,inner sep=2pt},
	fine/.style={circle,draw=black,inner sep=1pt}
]

\foreach \x in {0,...,3}
	\foreach \y in {0,...,7}
		\node[coarse] at (\x,\y){ };
\foreach \x in {4,...,7}
	\foreach \y in {0,...,2}
		\node[coarse] at (\x,\y){ };
\foreach \x in {4,...,7}
	\foreach \y in {5,...,7}
		\node[coarse] at (\x,\y){ };

\draw (9,3.5) node[cross out,draw=black,line width=0.5mm,rotate=45]{};

\foreach \x in {0,...,8}
	\foreach \y in {0,...,6}
		\node[fine] at (11+0.5*\x,2+0.5*\y){ };

\draw[->,line width=0.5mm] (17.5,3.5) -- (19.5,3.5);

\foreach \x in {22,...,25}
	\foreach \y in {0,...,7}
		\node[coarse] at (\x,\y){ };
\foreach \x in {26,...,29}
	\foreach \y in {0,...,2}
		\node[coarse] at (\x,\y){ };
\foreach \x in {26,...,29}
	\foreach \y in {5,...,7}
		\node[coarse] at (\x,\y){ };

\foreach \x in {0,...,8}
	\foreach \y in {0,...,6}
		\node[fine] at (25+0.5*\x,2+0.5*\y){ };
\end{tikzpicture}

\caption{Teiligitter in der Multi-Domain Herangehensweise}
\end{figure}

\noindent
Vorteil gegenüber des Multi-Grid Ansatzes ist hier der weiter reduzierte Speicherbedarf sowie erwartete Einsparungen in der benötigten Rechenzeit. Erkauft werden diese Vorteile durch aufwendigere Kopplung \cite[Kap.~3.1]{lagrava12} der verschiedenen Teilgitter in den Übergangsbereichen.

\subsubsection{Koinzidierende und versetzte Gitter}

Neben der Unterscheidung zwischen Multi-Grid und Multi-Domain Ansätzen kann auch die Positionierung des groben im Bezug auf das feine Gitter varieren. Gitterkonstellationen, in denen grobe Knoten soweit möglich mit feinen Knoten übereinstimmen -- wie auch in den beiden vorherigen Abbildung zu sehen -- werden als koinzidierend bezeichnet:

\begin{figure}[h]
\centering
\begin{tikzpicture}[
	scale=1.0,
	fine/.style={circle,draw=black,thick,inner sep=2},
]

\draw[step=1.0,black] (0.75,0.75) grid (6.25,6.25);
\draw[step=0.5,gray,thin] (2,2) grid (5,5);

\foreach \x in {0,...,6}
	\foreach \y in {0,...,6}
		\node[fine] at (2+0.5*\x,2+0.5*\y){ };
\end{tikzpicture}

\caption{Koinzidierende Gitter}
\label{fig:CoincidingGrid}
\end{figure}

\noindent
Lösen wir uns von dieser Einschränkung und positionieren das feine Gitter abseits der groben Gitterknoten, sprechen wir über zueinander versetzte -- im angelsächsischen Sprachraum als \emph{staggered} beschriebene -- Gitter:

\begin{figure}[h]
\centering
\begin{tikzpicture}[
	scale=1.0,
	fine/.style={circle,draw=black,thick,inner sep=2},
]

\draw[step=1.0,black] (0.75,0.75) grid (6.25,6.25);
\draw[step=0.5,gray,thin,xshift=1.75cm,yshift=1.75cm] (0,0) grid (3.5,3.5);

\foreach \x in {0,...,7}
	\foreach \y in {0,...,7}
		\node[fine] at (1.75+0.5*\x,1.75+0.5*\y){ };
\end{tikzpicture}

\caption{Zueinander versetzte Gitter ohne überlappende Knoten}
\label{fig:StaggeredGrid}
\end{figure}

\newpage
\section{Verfeinerungsmethode nach Lagrava et al.}

Wie in Kapitel~\ref{sec:olbRefinementChoice} noch näher begründet werden wird, bieten sich der Multi-Domain Ansatz als Grundlage für Gitterverfeinerung in OpenLB an. Passend zu dieser Wahl sowie der, im Rahmen dieser Arbeit getroffenen, Einschränkung auf zweidimensionale LBM mit BGK-Kollisionsoperator haben Lagrava et al. 2012 in \citetitle{lagrava12}~\cite{lagrava12} ein solches Verfeinerungsverfahren entwickelt. Die Stuktur dieses Verfahrens, mit potenziell austauschbaren Restriktions- und Interpolationsverfahren im zentralen Kopplungsschritt, erscheint dabei sogleich auch als Fundament eines Multi-Domain Gitterverfeinerungsframeworks in OpenLB.

\subsection{Übersicht}

Das Verfahren basiert auf dem Multi-Domain Ansatz \cite[Kap.~3.1]{lagrava12} mit koinzidierenden Gittern. Es werden also die feiner aufgelösten Teilbereiche der Simulationsdomäne so aus dem gröber aufgelösten Gitter ausgeschlossen, dass sie sich nur in Übergangsbereichen überlappen.

\begin{figure}[h]
\centering
\begin{tikzpicture}[
	scale=0.4,
	coarse/.style={circle,draw=black,inner sep=2pt},
	fine/.style={circle,draw=black,inner sep=1pt}
]

\foreach \x in {0,...,3}
	\foreach \y in {0,...,7}
		\node[coarse] at (\x,\y){ };
\foreach \x in {4,...,7}
	\foreach \y in {0,...,2}
		\node[coarse] at (\x,\y){ };
\foreach \x in {4,...,7}
	\foreach \y in {5,...,7}
		\node[coarse] at (\x,\y){ };

\draw (9,3.5) node[cross out,draw=black,line width=0.5mm,rotate=45]{};

\foreach \x in {0,...,10}
	\foreach \y in {0,...,10}
		\node[fine] at (11+0.5*\x,1+0.5*\y){ };

\draw[->,line width=0.5mm] (18,3.5) -- (20,3.5);

\foreach \x in {22,...,25}
	\foreach \y in {0,...,7}
		\node[coarse] at (\x,\y){ };
\foreach \x in {26,...,29}
	\foreach \y in {0,...,2}
		\node[coarse] at (\x,\y){ };
\foreach \x in {26,...,29}
	\foreach \y in {5,...,7}
		\node[coarse] at (\x,\y){ };

\foreach \x in {0,...,10}
	\foreach \y in {0,...,10}
		\node[fine] at (24+0.5*\x,1+0.5*\y){ };
\end{tikzpicture}

\caption{Multi-Domain Herangehensweise mit Übergangsbereich \cite[vgl.~Abb.~3]{lagrava12}}
\label{fig:MultiDomainOverlap}
\end{figure}

\noindent
In diesen Übergangsbereichen, welche eine Breite von mindestens einer Einheit des gröberen Zellabstands haben, liegt die Hauptarbeit des Verfeinerungsverfahrens.

\bigskip
Während der Übergang vom feinen zum groben Gitter sich im Wesentlichen auf eine skalierte und gefilterte Restriktion der Verteilungen beschränkt, gestaltet sich der Übergang vom groben zum feinen Gitter aufwendiger, da feine Knoten, für deren Position kein grober Knoten existiert, aus den übrigen Daten interpoliert werden müssen.

Entsprechend liegt der Fokus des von Lagrava et al. entwickelten Algorithmus auf der Auswahl des Interpolationsverfahrens sowie der Skalierung der physikalischen Werte zwischen den unterschiedlich aufgelösten Verteilungen. Um diese Kopplung der verschiedenen Gitterauflösungen theoretisch erfassen zu können, müssen wir zunächst die Gitter selbst konkreter definieren.

\newpage
\subsection{Gitterdiskretisierung}

\begin{Definition}[Grobe und feine Simulationsdomänen]
\label{def:SimDomain}
Sei \(D \subseteq \R^2\) die physikalische Simulationsdomäne unabhängig der verwendeten Gitterauflösung. Diese Teilmenge von \(\R^2\) sei dabei bereits in Hinblick auf die angestrebte kartesische Diskretisierung gewählt, d.h. als Vereinigung zusammenhängender Rechtecke.

Den durch ein grobes Gitter abzubildenden Teilbereich bezeichnen wir mit \(D_g \subset D\). Weiter ist \(D_f \subset D\) mit \(D_f \cap D_g \neq \emptyset\) der zu verfeinernde Teilbereich. O.B.d.A. nehmen wir an, dass \(D_g \cup D_f = D\) gilt, die Simulationsdomäne also in genau zwei Auflösungsstufen aufgeteilt wird.
\end{Definition}

\begin{figure}[h]
\centering
\begin{tikzpicture}[
	scale=0.9,
	x={(-10:1cm)},y={(220:1cm)},z={(90:1cm)},
	coarse/.style={circle,draw=black,inner sep=2},
	fine/.style={cross out,draw=black,inner sep=1},
]

\draw[opacity=0.5,dotted] (0,0,-2) -- (0,0,0);
\draw[opacity=0.5,dotted] (0,5,-2) -- (0,5,0);
\draw[opacity=0.5,dotted] (5,0,-2) -- (5,0,0);
\draw[opacity=0.5,dotted] (5,5,-2) -- (5,5,0);

\draw[opacity=0.5,dotted] (1,1,-2) -- (1,1,4);
\draw[opacity=0.5,dotted] (1,4,-2) -- (1,4,4);
\draw[opacity=0.5,dotted] (4,1,-2) -- (4,1,4);
\draw[opacity=0.5,dotted] (4,4,-2) -- (4,4,4);

\draw[opacity=0.5,dotted] (1.5,1.5,-2) -- (1.5,1.5,2);
\draw[opacity=0.5,dotted] (1.5,3.5,-2) -- (1.5,3.5,2);
\draw[opacity=0.5,dotted] (3.5,1.5,-2) -- (3.5,1.5,2);
\draw[opacity=0.5,dotted] (3.5,3.5,-2) -- (3.5,3.5,2);

\begin{scope}[xyp=-2]
\foreach \x in {0,...,10}
	\foreach \y in {0,...,10}{
		\ifthenelse{\x>6 \OR \x<4 \OR \y>6 \OR \y<4}{
			\node[coarse] at (0.5*\x,0.5*\y){ };
		}{}
	}

\foreach \x in {0,...,12}
	\foreach \y in {0,...,12}{
		\node[fine] at (1+0.25*\x,1+0.25*\y){ };
	}

\node[right] at (6,0) {Gitter \(\G\) und \(\F\)};
\end{scope}

\begin{scope}[xyp=0]
\fill[opacity=0.6,gray,even odd rule] (0,0) rectangle (5,5) (1.5,1.5) rectangle (3.5,3.5);
\draw[thick,even odd rule] (0,0) rectangle (5,5) (1.5,1.5) rectangle (3.5,3.5);
\draw[dashed] (1,1) rectangle (4,4);
\draw[pattern=north east lines,even odd rule] (1,1) rectangle (4,4) (1.5,1.5) rectangle (3.5,3.5);

\node[right] at (6,0) {Grobe Domäne \(D_g\)};
\end{scope}

\begin{scope}[xyp=2]
\fill[thick,opacity=0.6,gray,even odd rule] (1,1) rectangle (4,4) (1.5,1.5) rectangle (3.5,3.5);
\draw[thick] (1,1) rectangle (4,4) (1.5,1.5) rectangle (3.5,3.5);

\node[right] at (6,0) {Übergangsbereich \(D_g \cap D_f\)};
\end{scope}

\begin{scope}[xyp=4]
\draw[thick] (1,1) rectangle (4,4);
\fill[opacity=0.6,gray] (1,1) rectangle (4,4);

\node[right] at (6,0) {Feine Domäne \(D_f\)};
\end{scope}

\end{tikzpicture}

\caption{Schematische Darstellung der Simulationsdomänen der Gitter}
\label{fig:SimDomain}
\end{figure}

Die Annahme von zwei Verfeinerungsstufen ist berechtigt, da mehrfach verfeinerte Gitter sich durch erneute -- rekursive -- Anwendung von Definition~\ref{def:SimDomain} mit \(\tilde{D} := D_f\) modellieren lassen. Auch Verfeinerung eines groben Gitters in mehreren disjunkten Bereichen kann unter dieser Annahme betrachtet werden, da für die Koppelung zwischen Gittern nur Knoten in \(D_f\) relevant sind.

\begin{Definition}[Diskretisierung der Simulationsdomänen]
\label{def:DiskretRefinedGitter}
Wir betrachten kartesische Gitter \(\G\) und \(\F\) als Diskretisierungen der Simulationsdomänen \(D_g\) bzw. \(D_f\). Diese seien so gewählt, dass sie gerade die konvexen Hüllen ihrer koinzidierenden Diskretisierungsgitter beschreiben.
\begin{align*}
\G &\subset D_g \cap \{ x \in \R^2 | \exists i \in \Z^2 : x = \delta x_g \cdot i \} && \text{Gröberes Gitter} \\
\F &\subset D_f \cap \{ x \in \R^2 | \exists i \in \Z^2 : x = \delta x_f \cdot i \} && \text{Feineres Gitter}
\end{align*}
\(\delta x_g = \delta x_g / 2 \in \R_+\) seien die Diskretisierungsauflösungen im Verhältnis \(1:2\).
\end{Definition}

Zur Entwicklung der Gitterkopplung fordern wir, dass sich \(\G\) und \(\F\) um mindestens eine grobe Gitterweite \(\delta x_g\) überlappen -- vgl. Abbildungen \ref{fig:MultiDomainOverlap} und \ref{fig:OverlapZone}. Die Seitenlängen der konvexen Hüllen \(D_g\) und \(D_f\) sind ganzzahlige Vielfache von \(\delta x_g\) und \(\delta x_f\). Formal sei dabei das Innere der groben Domäne \(D_g\) ohne den Übergangsbereich der Breite \(\delta x_g\) gegeben als:
\[ D_g^\circ := \{ x \in D_g | \forall y \in \R^2 \setminus D_g : \|x-y\| > \delta x_g \} \]
Vergleiche hierzu den unschraffierten Bereich der Darstellung von \(D_g\) in Abbildung~\ref{fig:SimDomain}.
Unter der Annahme, dass \(D_g\) die feine Simulationsdomäne \(D_f\) komplett umschließt, also der komplette Rand des feinen Gitters mit dem groben gekoppelt werden soll, fordern wir dann für den Rand des feinen Gitters:
\[ \partial D_f \stackrel{!}{\subset} \partial D_g^\circ \numberthis\label{eq:FineBorderIntersectCoarseGrid} \]
Diese Einschränkung garantiert, dass die äußersten Gitterknoten des feinen Gitters sich soweit möglich mit groben Gitterknoten schneiden und nicht etwa zwischen zwei groben \emph{Gitterreihen} liegen.

\begin{figure}[h]
\centering
\begin{tikzpicture}[
	scale=1.5,
	coarse/.style={circle,draw=black,thick,inner sep=4},
	fine/.style={cross out,draw=black,thick,inner sep=2},
	arrow/.style={-{Latex[length=2mm]},thick}
]

\foreach \x in {-1,...,3}
	\foreach \y in {0,...,2}
		\node[coarse] at (\x,\y){ };

\foreach \x in {0,...,8}
	\foreach \y in {0,...,4}
		\node[fine] at (1.5+\x*0.5,\y*0.5){ };

\draw[arrow] (1.5,2.8) node[right] {Randknoten in \(\partial D_f\) für \(\partial D_f \not\subset \partial D_g^\circ\)} -- (1.5,2.3);
\draw[dashed,thick,pattern=north west lines, pattern color=red] (1.3,-0.2) rectangle (1.7,2.2);
\end{tikzpicture}

\caption{Mit Forderung (\ref{eq:FineBorderIntersectCoarseGrid}) ausgeschlossener Übergangsbereich \cite[vgl. Abb.~9]{lagrava12}}
\label{fig:InvalidOverlapArea}
\end{figure}

\noindent
Wir können die Gitterknoten der Übergangsbereiche nun detailliert klassifizieren:

\begin{figure}[h]
\centering
\begin{tikzpicture}[
	scale=1.5,
	coarse/.style={circle,draw=black,thick,inner sep=4pt},
	fine/.style={cross out,draw=black,thick,minimum size=8pt},
	arrow/.style={-{Latex[length=2mm]},thick}
]

\foreach \x in {-1,...,3}
	\foreach \y in {0,...,2}
		\node[coarse] at (\x,\y){ };

\foreach \x in {0,...,8}
	\foreach \y in {0,...,4}
		\node[fine] at (2+\x*0.5,\y*0.5){ };

\draw[dashed,thick] (1.8,-0.2) rectangle (2.2,2.2);
\draw[dashed,thick] (2.8,-0.2) rectangle (3.2,2.2);

\draw[arrow] (2,3.3) node[right] {Übertragung von grob nach fein, \(\U_{g \to f}\)} -- (2,2.3);
\draw[arrow] (3,2.8) node[right] {Übertragung von fein nach grob, \(\U_{f \to g}\)} -- (3,2.3);

\draw[decorate,decoration={brace,amplitude=10pt,mirror},line width=1pt]
(1.8,-0.4) -- (3.2,-0.4) node[midway,below,yshift=-6pt] {Übergangsbereich};
\end{tikzpicture}

\caption{Skizze des Übergangsbereich \cite[vgl.~Abb.~4]{lagrava12}}
\label{fig:OverlapZone}
\end{figure}

\begin{Definition}[Gitterknoten der Übergangsbereiche]
\label{def:OverlapGridNodes}
\begin{align*}
\U_g &:= \G \cap \F && \text{Grobe Knoten im Übergangsbereich} \\
\U_f &:= \F \cap (D_f \cap D_g) && \text{Feine Knoten im Übergangsbereich} \\
\U_{g \to f} &:= \partial D_f \cap \U_f && \text{Knoten mit Übertragung von grob nach fein} \\
\U_{f \to g} &:= \partial D_g \cap \U_g && \text{Knoten mit Übertragung von fein nach grob}
\end{align*}
\end{Definition}

Die Übertragungsrichtungen in \(\U_{g \to f}\) und \(\U_{f \to g}\) ergeben sich aus den jeweils fehlenden Verteilungsfunktionen an den Rändern der Gitter. So fehlen beispielsweise zur Kollision der groben Gitterknoten in \(\U_{f \to g}\) Verteilungsfunktionen in Richtung des feinen Gitters, während die feinen Zellen in dieser Menge noch vollständig definiert sind, da sie im Inneren des feinen Gitters liegen.

Mit diesem Argument lässt sich auch die Notwendigkeit eines Übergangsbereiches \(\U_g \cup \U_f\) der Mindestbreite \(\delta x_g\) erklären: Gäbe es diesen nicht, so fehlten an der Grenze zwischen grobem und feinem Gitter Verteilungsfunktionen in beide Richtungen zugleich.

\bigskip
Anders als noch in Definition~\ref{def:SpatialDiscretizationLBM} betrachten wir die Gitter jetzt also nicht mehr unabhängig des darzustellenden physikalischen Modells, sondern unterscheiden anhand der physikalischen Auflösung \(\delta x\). Während, im Kontext der LBM an sich, weiterhin für beide Gitter \(\Delta x = 1\) gesetzt wird, führt die Relation von \(\delta x_g\) und \(\delta x_f\) im kommenden Kapitel~\ref{kap:Skalierung} u.a. zu einer Relation zwischen grober und feiner Relaxionszeit.

Eine stringente Behandlung von Gitterkopplung in diesem Modell benötigt Abbildungen der physikalisch eingebetteten Knoten aus \(\G\) und \(\F\) in die zugehörigen \emph{implementierenden} Gitter mit uniformer Auflösung \(\Delta x = 1\). Diese stellen gerade die Definitionsmengen der groben bzw. feinen Verteilungsfunktionen dar.

\begin{Definition}[Abbildung auf implementierende Gitter]
\label{def:BijImplGitter}
Sei \(\# \in \{f, g\}\) Symbol des feinen oder groben Gitters.
Dann können wir o.B.d.A. einen beliebigen physikalischen Knoten \(x_{0,\#}^\text{phys}\) mit dem Knoten \(x_0^\text{impl} = 0 \in L\) identifizieren. Eine Bijektion zwischen physikalischen und implementierenden Gittern ist damit schon eindeutig definiert:
\begin{align*}
x_\#^\text{impl}(x^\text{phys}) &:= \frac{1}{\delta x_\#} (x^\text{phys} - x_{0,\#}^\text{phys}) \\
x_\#^\text{phys}(x^\text{impl}) &:= x_{0,\#}^\text{phys} + \delta x_\# \cdot x^\text{impl}
\end{align*}
Diese Abbildung der physikalisch eingebetteten Gitterknoten in die Definitionsmenge der Verteilungsfunktionen nehmen wir dabei zur Vereinfachung der Notation implizit an, wann immer die Verteilung in Elementen aus \(\G\) oder \(\F\) betrachtet wird.
\end{Definition}

Da die Einbettungen von Knoten verschiedener Auflösungen in deren Übergangsbereichen nicht disjunkt sind, benötigen wir im Weiteren gitterspezifische Bezeichnungen für die Verteilungen und deren Momente:

\begin{Definition}[Gitterspezifische Verteilungsfunktionen und Momente]
Sei \(\# \in \{f, g\}\) wieder Symbol des feinen oder groben Gitters. Wir bezeichnen dann mit \(f_{\#,i}\) die \(i\)-te Verteilungsfunktion des entspechenden Gitters. Analog formulieren sich mit Definition~\ref{def:Momente} die Momente für \(x \in \G\) respektive \(x \in \F\):
\begin{align*}
\rho_\#(x) &:= \sum_{j=0}^{q-1} f_{\#,j}(x) \\
u_\#(x) &:= \frac{1}{\rho_\#(x)} \sum_{j=0}^{q-1} \xi_j f_{\#,j}(x)
\end{align*}
\end{Definition}

\bigskip
Zusammenfassend wird die Aufgabe der im kommenden Kapitel zu erarbeitenden Skalierungs-, Restriktions- und Interpolationsschritte also \emph{nur} darin bestehen, die jeweils fehlenden Verteilungsfunktionen möglichst gut zu rekonstruieren.

\newpage
\subsection{Komponenten der Gitterkopplung}
\subsubsection{Skalierung}
\label{kap:Skalierung}

Während die Skalierung räumlicher Größen durch die Festlegung des Verfahrens auf Übergänge im Verhältnis \(1:2\) definiert ist, eröffnen sich für die zeitliche Skalierung zwei Möglichkeiten: Konvektive oder diffusive Skalierung. Unterschied der beiden Ansätze ist dabei das jeweilige Verhältnis zwischen räumlicher und zeitlicher Auflösung.

\begin{Definition}[Konvektive Skalierung]
Sei \(\delta t > 0\) die zeitliche und \(\delta x > 0\) die räumliche Diskretisierung. Dann gilt bei konvektiver Skalierung das Verhältnis:
\[ \delta t \sim \delta x \]
Es besteht also eine lineare Proportionalität.
\end{Definition}

\begin{Definition}[Diffusive Skalierung]
Sei \(\delta t > 0\) die zeitliche und \(\delta x > 0\) die räumliche Diskretisierung. Dann gilt bei diffusiver Skalierung das Verhältnis:
\[ \delta t \sim \delta x^2 \]
Es besteht hier also eine quadratische Proportionalität. Im Vergleich zu einer konvektiven Skalierung ist die zeitliche Auflösung somit um eine Ordnung feiner.
\end{Definition}

Es ist klar zu erkennen, dass diffusive Skalierung einen deutlich größeren numerischen Aufwand gegenüber der konvektiven Skalierung nach sich zieht. Vorteil der bei diffusiver Skalierung erhöhten Schrittanzahl pro Zeiteinheit sind kleinere Fehler.

Für die Autoren des hier erörterten Gitterverfeinerungsverfahrens überwog jedoch das Argument der numerischen Effizienz, weshalb auch wir hier nun die konvektive Skalierung betrachten wollen. Die Austauschbarkeit des Skalierungsverfahrens sollte jedoch bei der Implementierung in OpenLB beachtet werden, da dieser Aspekt eine weitere prinzipiell flexible Komponente des Verfahrens darstellt.

\bigskip

Aus der Wahl von konvektiver Skalierung ergibt sich zunächst:
\[\frac{\delta t_g}{\delta x_g} = \frac{\delta t_f}{\delta x_f} \land \delta x_f = \frac{\delta x_g}{2} \implies \delta t_f = \frac{\delta t_g}{2} \numberthis\label{eq:gridTime}\]
Auf dem feinen Gitter müssen also doppelt so viele Zeitschritte wie auf dem groben Gitter durchgeführt werden. Geschwindigkeit, Druck und Dichte sind stetig im Gitterübergang. Dies gilt nicht für die kinetische Viskosität \(\nu = c_s^2 \tau\), was wir bei der Herleitung der feinen Relaxionszeit \(\tau_f\) aus \(\tau_g\) beachten müssen.

\begin{Definition}[Physikalische Reynolds-Zahl]
\label{def:PhysicalReynoldsNumber}
Seien \(U\) die charakteristische Geschwindigkeit, \(L\) die charakteristische Länge und \(\nu\) die kinetische Viskosität in physikalischen Einheiten. Dann ist die Reynolds-Zahl definiert als: \[\text{Re} := \frac{U L}{\nu}\]
\end{Definition}

\begin{Definition}[Lattice Reynolds-Zahl]
\label{def:LatticeReynoldsNumber}
Sei \(\# \in \{f, g\}\) Symbol des feinen oder groben Gitters.
Seien \(U_\# := \delta t_\# / \delta x_\# \cdot U\) die charakteristische Geschwindigkeit, \(L_\# := 1 / \delta x_\# \cdot L\) die charakteristische Länge und \(\nu_\# := c_s^2 \tau_\#\) die kinetische Viskosität in Lattice-Einheiten. Dann ist die \emph{Lattice} Reynolds-Zahl des feinen bzw. groben Gitters definiert als: \[ \text{Re}_\# := \frac{U_\# L_\#}{\nu_\#} = \frac{\delta t_\# U L}{(\delta x_\#)^2 \nu_\#} \]
\end{Definition}

Wir erzwingen nun mit \(\text{Re}_g = \text{Re}_f\) die Unabhängigkeit von Reynolds-Zahl und Gitterauflösung. Diese Gleichsetzung ist sinnvoll, da die Reynolds-Zahl gerade die Vergleichbarkeit von Strömungen verschiedener Modellgrößen ermöglicht. Durch Einsetzen von Definition~\ref{def:LatticeReynoldsNumber} erhalten wir eine Verknüpfung der Relaxionszeiten \(\tau_f\) und \(\tau_g\):
\begin{align*}
\text{Re}_g = \text{Re}_f &\iff \frac{\delta t_g U L}{(\delta x_g)^2 \nu_g} = \frac{\delta t_f U L}{(\delta x_f)^2 \nu_f} \\
&\iff \frac{\delta t_g}{(\delta x_g)^2 \nu_g} = \frac{\delta t_f}{(\delta x_f)^2 \nu_f} \\
&\iff \frac{\delta t_g}{(\delta x_g)^2 c_s^2 \tau_g} = \frac{\delta t_f}{(\delta x_f)^2 c_s^2 \tau_f} \\
&\iff \tau_f = \frac{\delta t_f (\delta x_g)^2}{(\delta x_f)^2 \delta t_g} \tau_g \\
&\iff \tau_f = 2 \tau_g \numberthis\label{eq:gridTau}\\
\end{align*}

Für die zur expliziten Lösung der diskreten LBM BGK Gleichung in Definition~\ref{def:LBGKeq} verschobenen Relaxionszeiten ergibt sich somit:
\[\overline{\tau_f} = 2 \overline{\tau_g} - \frac{1}{2} \numberthis\label{eq:gridTauShift}\]

Die Equilibriumsverteilung \(f_i^\text{eq}\) ergibt sich nach Definition~\ref{def:fieq} aus Geschwindigkeit \(u\) und Dichte \(\rho\). Sie sind also explizit unabhängig der Gitterauflösung und, wie erwähnt, stetig im Gitterübergang. Diese Aussage gilt nicht für die Nicht-Equilibriumsverteilung \(f_i^\text{neq}\), da diese nach (\ref{eq:approxFneq}) von dem Geschwindigkeitsgradienten \(\nabla u\) abhängt.

Bezeichnen nun \(f_{f,i}^\text{neq}\) und \(f_{g,i}^\text{neq}\) die gitterspezifischen Nicht-Equilibriumanteile und \(\mathrm{S}_f\) sowie \(\mathrm{S}_g\) die entsprechenden Verzerrungstensoren. Zur Skalierung von \(f_{f,i}^\text{neq}\) suchen wir ein \(\alpha \in \R\) s.d. gilt: \[f_{f,i}^\text{neq} = \alpha f_{g,i}^\text{neq} \numberthis\label{eq:scaleFneqReq}\]
Mit Hilfe von (\ref{eq:approxFneq}) lässt sich diese Relation nach \(\alpha\) auflösen:
\begin{align*}
f_{f,i}^\text{neq} = \alpha f_{g,i}^\text{neq} &\iff -\frac{w_i \rho \overline{\tau_f}}{c_s^2} \mathrm{Q}_i : \mathrm{S}_f = -\alpha \left( \frac{w_i \rho \overline{\tau_g}}{c_s^2} \mathrm{Q}_i : \mathrm{S}_g \right) \\
&\iff \overline{\tau_f} \mathrm{Q}_i : \mathrm{S}_f = \alpha \overline{\tau_g} \mathrm{Q}_i : \mathrm{S}_g \\
&\iff \overline{\tau_f} \delta t_f \mathrm{Q}_i : \mathrm{S} = \alpha \overline{\tau_g} \delta t_g \mathrm{Q}_i : \mathrm{S} \\
&\iff \alpha = \frac{\delta t_f}{\delta t_g} \frac{\overline{\tau_f}}{\overline{\tau_g}}\\
\end{align*}
Einsetzen der Relationen (\ref{eq:gridTime}) und (\ref{eq:gridTauShift}) reduziert den Skalierungsfaktor auf einen nur von der groben Relaxionszeit abhängigen Ausdruck:
\begin{align*}
\alpha &= \frac{\delta t_f}{\delta t_g} \frac{\overline{\tau_f}}{\overline{\tau_g}} \\
&= \frac{1}{2} \frac{2\overline{\tau_g} - \frac{1}{2}}{\overline{\tau_g}} \\
&= 1 - \frac{1}{4\overline{\tau_g}} \numberthis\label{eq:scaleFactor}
\end{align*}
Schließlich erhalten wir so die folgende Relation der Nicht-Equilibriumsverteilungen:
\[f_{f,i}^\text{neq} = \left( 1-\frac{1}{4\overline{\tau_g}} \right) f_{g,i}^\text{neq} \numberthis\label{eq:scaleFneq}\]

Insgesamt haben wir hiermit die Skalierung der Diskretisierungen in Raum und Zeit, der Relaxionszeit sowie der Nicht-Equilibriumsverteilung zwischen den Gittern \(\F\) und \(\G\) hergeleitet.

\bigskip

Seien \(x_{f \to g} \in \U_{f \to g}\) und \(x_{g \to f} \in \U_{g \to f}\) die Knoten aus dem Übergangsbereich mit Übertragung von fein nach grob bzw. von grob nach fein. Dann gelten bei Erinnerung an die implizite Knotenabbildung~\ref{def:BijImplGitter}:
\begin{align}
f_{g,i}(x_{f \to g}) &= f_i^\text{eq}(\rho(x_{f \to g}), u(x_{f \to g})) + \left(1-\frac{1}{4\overline{\tau_g}}\right)^{-1} f_{f,i}^\text{neq}(x_{f \to g}) \label{eq:basicF2G} \\
f_{f,i}(x_{g \to f}) &= f_i^\text{eq}(\rho(x_{g \to f}), u(x_{g \to f})) + \left(1-\frac{1}{4\overline{\tau_g}}\right) f_{g,i}^\text{neq}(x_{g \to f}) \label{eq:basicG2F}
\end{align}

Die zusammengesetzten Verteilungsfunktionen von Übergangsknoten des einen Gitters lassen sich also durch Skalierung des Nicht-Equilibriumanteils der Verteilungsfunktionen des jeweils anderen Gitters rekonstruieren. Leider reicht dies noch nicht zur vollständigen Beschreibung eines Gitterverfeinerungsverfahrens, da nicht für alle feinen Gitterknoten im Übergangsbereich passende grobe Gitterknoten existieren -- vgl. dazu Abbildung~\ref{fig:OverlapZone}. Auch der Übergang von fein nach grob gestaltet sich trotz passenden feinen Knoten potenziell komplizierter, als eine bloße Skalierung, wie wir im nächsten Kapitel zeigen werden.

\newpage
\subsubsection{Restriktion}

Kraft seiner höheren Auflösung enthält das feine Gitter mehr Informationen als das umgebende grobe Gitter. Der Übergang von fein nach grob stellt also eine Restriktion der Verteilungsfunktionen dar.

Konkret suchen wir nach einer sinnvollen Definition der in \(x_{f \to g} \in \U_{f \to g}\) fehlenden Verteilungsfunktionen \(f_{g,i}\). Eine Solche ergibt sich aus der skalierten Dekomposition \ref{eq:basicF2G} durch Ersetzen der einfachen Nicht-Equilibriumsverteilung \(f_{f,i}^\text{neq}(x_{f \to g})\) mit einer restringierten Variante ebendieser.
\[f_{g,i}(x_{f \to g}) = f_i^\text{eq}(\rho(x_{f \to g}), u(x_{f \to g})) + \alpha^{-1} \ \resarg{i}{x_{f \to g}} \numberthis\label{eq:restrictedF2G}\]
Wir bemerken an dieser Stelle, dass nur die Nicht-Equilibriumsverteilung durch eine Restriktionsoperation eingeschränkt wird, während der Equilibriumanteil unangetastet bleibt. Dies ist damit zu begründen, dass Dichte und Geschwindigkeit bei der von uns verwendeten konvektiven Skalierung im Gitterübergang stetig bleiben.

Die skalierte Dekomposition \ref{eq:basicF2G} lässt sich in der Schreibweise von \ref{eq:restrictedF2G} formulieren, wenn die Identität als Restriktionsoperation eingesetzt wird: \[\resarg{i}{x_{f \to g}} := f_{f,i}^\text{neq}(x_{f \to g})\]

Die für unser Verfahren \cite[Kap.~3.3]{lagrava12} beschriebene Restriktion ist der Mittelwert aller umliegenden gerichteten Nicht-Equilibriumanteilen:
\[\resarg{i}{x_{f \to g}} := \frac{1}{q} \sum_{j=0}^{q-1} f_{f,i}^\text{neq}(x_{f \to g} + \delta x_f \xi_j) \numberthis\label{eq:neqAvgRestrictionF2G}\]

\newpage
\subsubsection{Interpolation}\label{kap:Interpolation}

Zunächst ergänzen wir die Gitterteilmengen aus Definition~\ref{def:OverlapGridNodes} um eine Unterscheidung zwischen alleinstehenden feinen Knoten und solchen für die ein, der Übertragung von grob nach fein dienlicher, grober Knoten existiert.
\begin{Definition}[Gitterknoten mit Übertragung von grob nach fein]
\begin{align*}
\U_{g \to f}^g &:= \U_{g \to f} \cap \U_g && \text{Doppelte Knoten mit Übertragung von grob nach fein} \\
\U_{g \to f}^f &:= \U_{g \to f} \setminus \U_g && \text{Alleinstehende feine Knoten mit Übertragung von grob nach fein}
\end{align*}
\end{Definition}

Für \(x_{g \to f}^g \in \U_{g \to f}^g\) gilt insbesondere \(x_{g \to f}^g \in \U_g \cap \, \U_f\). Es existieren in diesen Gitterpunkten also vollständig definierte grobe Verteilungsfunktionen, die wir zur Bestimmung der Momente \(\rho\) und \(u\) in (\ref{eq:basicG2F}) heranziehen können. Entsprechend besitzen wir zur Interpolation des gesuchten Wertes geschickterweise eine Stützstelle an eben dessen Position. Die \emph{Interpolation} von \(x_{g \to f}^g\) beschränkt sich folglich auf die Skalierung~(\ref{eq:basicG2F}):
\[f_{f,i}(x_{g \to f}^g) = f_i^\text{eq}(\rho_g(x_{g \to f}^g), u_g(x_{g \to f}^g)) + \alpha f_{g,i}^\text{neq}(x_{g \to f}^g) \numberthis\label{eq:expandedDirectG2F}\]

Für \(x_{g \to f}^f \in \U_{g \to f}^f\) gilt insbesondere \(x_{g \to f}^f \notin \U_g\). Es existieren in diesen Gitterpunkten also im Gegensatz zur Situation \ref{eq:expandedDirectG2F} keine groben Verteilungsfunktionen. Die fehlenden Werte zur Bestimmung der Momente sowie des Nicht-Equilibriumanteils in (\ref{eq:basicG2F}) müssen hier also aus den umliegenden groben Verteilungsfunktionen interpoliert werden:
\[f_{f,i}(x_{g \to f}^f) = f_i^\text{eq}(\ipolarg{\rho_g}{x_{g \to f}^f}, \ipolarg{u_g}{x_{g \to f}^f}) + \alpha \ipolarg{f_{g,i}^\text{neq}}{x_{g \to g}^f} \numberthis\label{eq:expandedInterpolG2F}\]
Die unbekannten Werte der Moment- und Nicht-Equilibriumfunktionen werden in diesem Ausdruck durch eine Interpolationsoperation \(\ipol\) genähert. Neben der Art der Restriktion \(\res\) stellt die Wahl des Interpolationsverfahrens einen weiteren zentralen und flexiblen Bestandteil des, auf diesen Seiten nachvollzogenen, Gitterverfeinerungsverfahren dar.

\bigskip

Stützstellen für die Interpolation seien hier die parallel zum Gitterübergang liegenden groben Nachbarknoten des gesuchten Punktes \(x_{g \to f}^f\). Wir adressieren diese, parallel zu einem Einheitsvektor \(v\) positionierten, Stützknoten mit:
\[\mathcal{N}_{x_{g \to f}^f} := \left\{x \in \G \middle| x = x_{g \to f}^f + j \, \delta x_f \, v,\ j \in \Z \right\} \subseteq \U_{g \to f}^g\]
Bekannte Stützwerte von \(\ipolarg{\star}{x}\) befinden sich also relativ zum gesuchten Knoten \(x_{g \to f}^f\) in, um ungerade Vielfache der feinen Schrittweite \(\delta x_f\) skalierten, Verschiebungen entlang der normierten Übergangsparallele \(v\) (vgl. Abbildungen \ref{fig:InterpolationBasis} und \ref{fig:InterpolationDetail}). Die Einschränkung der hinzugezogenen Stützen auf Knoten, welche parallel zum Übergang liegen, wurde von Lagrava et al. so gewählt~\cite[Kap.~3.6]{lagrava12}, um in 2D ein eindimensionales Interpolationsproblem zu erhalten. Prinzipiell spricht nichts gegen eine Einbeziehung umfangreicherer Teilmengen der groben Knotennachbarschaft.

\begin{figure}[h]
\centering
\begin{tikzpicture}[
	scale=1.5,
	coarse/.style={circle,draw=gray,inner sep=4},
	fine/.style={cross out,draw=gray,inner sep=2},
	ibase/.style={coarse,draw=black,very thick},
	wantedfine/.style={fine,draw=black,ultra thick},
	arrow/.style={-{Latex[length=2mm]},thick}
]

\foreach \x in {-1,...,3}
	\foreach \y in {-1,...,2}
		\node[coarse] at (\x,\y){ };

\foreach \x in {0,...,8}
	\foreach \y in {-2,...,4}
		\node[fine] at (2+\x*0.5,\y*0.5){ };

\node[wantedfine] at (2,0.5){ };
\node[ibase] at (2,2){ };
\node[ibase] at (2,1){ };
\node[ibase] at (2,0){ };
\node[ibase] at (2,-1){ };

\draw[dashed,very thick] (1.8,-1.2) rectangle (2.2,2.2);
\draw[dashed,draw=gray] (2.8,-1.2) rectangle (3.2,2.2);

\draw[arrow] (2,3.3) node[right] {Interpolation von grob nach fein, \(\U_{g \to f}\)} -- (2,2.3);
\draw[arrow] (3,2.8) node[right] {Restriktion von fein nach grob, \(\U_{f \to g}\)} -- (3,2.3);

\draw[decorate,decoration={brace,amplitude=10pt,mirror},line width=1pt]
(1.8,-1.4) -- (3.2,-1.4) node[midway,below,yshift=-6pt] {Übergangsbereich};
\end{tikzpicture}

\caption{Stützstellen der Interpolation im Übergangsbereich}
\label{fig:InterpolationBasis}
\end{figure}

Um die kommenden Ausführungen auf das Wesentliche -- namentlich das Verfahren unabhängig der konkret zu interpolierenden Funktion -- zu konzentrieren, sei definiert:
\[\sipolarg{h} := \ipolarg{\star}{x_{g \to f}^f + h \, \delta x_f \, v} \text{ für Zielfunktion } \star \in \{\rho_g, u_g, f_{g,i}^\text{neq}\}\]
In dieser Formulierung suchen wir also eine möglichst gute Interpolation des Wertes in \(\sipolarg{0}\) anhand der Stützstellen \(\sipolarg{h}\) für kleine \(h \in \Z \setminus 2\Z\). Ein naheliegender Ansatz hierfür ist das arithmetische Mittel der beiden engsten Nachbarn:
\[\ipolarg{\star}{x_{g \to f}^f} = \sipolarg{0} = \frac{\sipolarg{-1} + \sipolarg{1}}{2} \numberthis\label{eq:ipol2ord}\]
Vorteil eines solch einfachen Verfahrens wäre, dass die benötigten groben Nachbarn auch an den Ecken des Übergangsbereiches existieren (vgl. Abbildung~\ref{fig:InterpolationEdgeCase}) und daher keine Sonderbehandlung erforderlich wird.
\begin{figure}[h]
\centering
\begin{tikzpicture}[
	scale=1.5,
	coarse/.style={circle,draw=gray,inner sep=4pt},
	ibase/.style={coarse,draw=black,very thick},
	fine/.style={cross out,draw=gray,minimum size=8pt},
	wantedfine/.style={fine,draw=black,ultra thick},
	arrow/.style={-{Latex[length=2mm]},thick}
]

\foreach \x in {-3.5,...,3.5}
	\node[coarse] at (\x,0){ };

\foreach \x in {-7,...,7}
	\node[fine] at (\x*0.5,0){ };

\node[wantedfine] at (0,0){ };
\node[ibase] at (-1.5,0){ };
\node[ibase] at (-0.5,0){ };
\node[ibase] at (0.5,0){ };
\node[ibase] at (1.5,0){ };

\draw[arrow] (0,0.8) node[right] {Gesuchter Wert \(\interpol{\star}{x_{g \to f}^f}\)} -- (0,0.3);
\draw[arrow] (-1.5,-0.8) node[below right] {Bekannte Werte \(\interpol{\star}{\U_{g \to f}^g}\)} -- (-1.5,-0.3);
\draw[arrow] (-0.5,-0.8) -- (-0.5,-0.3);
\draw[arrow] (0.5,-0.8) -- (0.5,-0.3);
\draw[arrow] (1.5,-0.8) -- (1.5,-0.3);

\end{tikzpicture}

\caption{Stützstellen der Interpolation im Detail}
\label{fig:InterpolationDetail}
\end{figure}

Bessere Näherungen können unter Einsatz weiterer Stützknoten erzielt werden. Wir berechnen dazu mit dem Schema der dividierten Differenzen die Faktoren der Newtonschen Interpolationsformel \cite[IV.3~(3.10)]{amann_escher} auf den in Abbildung~\ref{fig:InterpolationDetail} dargestellten Stützpunkten \((-3,\sipolarg{-3})\), \((-1,\sipolarg{-1})\), \((1,\sipolarg{1})\) und \((3,\sipolarg{3})\):
\begin{align*}
\sipolarg{x} :&= \sipolarg{-3} \\
&+ \frac{1}{2}(\sipolarg{-1}-\sipolarg{-3})(x+3)\\
&+ \frac{1}{8}(\sipolarg{1}-\sipolarg{-1}+\sipolarg{3})(x+3)(x+1) \\
&+ \frac{1}{48}(\sipolarg{3}-3\sipolarg{1}+3\sipolarg{-1}-\sipolarg{-3})(x+3)(x+1)(x-1)
\end{align*}
Ausgewertet in \(0\) erhalten wir folgenden Ausdruck als Näherung von \(\ipolarg{\star}{x_{g \to f}^f}\):
\[\sipolarg{0} = \frac{9}{16}(\sipolarg{-1} + \sipolarg{1}) - \frac{1}{16}(\sipolarg{-3} + \sipolarg{3}) \numberthis\label{eq:ipol4ord}\]
Hierbei handelt es sich um ein Verfahren vierter Ordnung, wie sich durch Einsetzen der Taylor-Entwicklung von \(\sipol\) um \(0\) in die Auswertung~(\ref{eq:ipol4ord}) zeigen lässt:
\begin{align*}
&\sipolarg{h} = \sipolarg{0} + \sipolderivarg{1}{0}h + \frac{1}{2}\sipolderivarg{2}{0}h^2 + \frac{1}{6}\sipolderivarg{3}{0}h^3 + \mathcal{O}(h^4) \numberthis\label{eq:sipolTaylorOrder4} \\
\implies &\frac{9}{16}(\sipolarg{-1} + \sipolarg{1}) - \frac{1}{16}(\sipolarg{-3} + \sipolarg{3}) \stackrel{(\ref{eq:sipolTaylorOrder4})}{=} \sipolarg{0} + \mathcal{O}(h^4)
\end{align*}

In Abbildung~\ref{fig:InterpolationEdgeCase} erkennen wir zwei mögliche Randfälle des Gitterübergangs, welche zu Problemen bei Nutzung des Interpolationsverfahren vierter Ordnung führen können.
\begin{figure}[h]
\centering
\begin{tikzpicture}[
	scale=1.5,
	coarse/.style={circle,draw=gray,inner sep=4pt},
	ibase/.style={coarse,draw=black!40!blue,very thick},
	fine/.style={cross out,draw=gray,minimum size=8pt},
	wantedfine/.style={fine,draw=black!40!blue,ultra thick},
	arrow/.style={-{Latex[length=2mm]},thick},
]

\foreach \x in {0,...,3}
	\foreach \y in {-2,...,2}
		\node[coarse] at (\x,\y){ };

\foreach \x in {4,...,6}
	\foreach \y in {0,...,2}
		\node[coarse] at (\x,\y){ };

\foreach \x in {0,...,8}
	\foreach \y in {-4,...,2}
		\node[fine] at (2+\x*0.5,\y*0.5){ };

\node[wantedfine] at (2,0.5){ };
\node[ibase] at (2,1){ };
\node[ibase] at (2,0){ };
\node[ibase] at (2,-1){ };

\node[wantedfine,draw=black!40!green] at (5.5,1){ };
\node[ibase,draw=black!40!green] at (6,1){ };
\node[ibase,draw=black!40!green] at (5,1){ };
\node[ibase,draw=black!40!green] at (4,1){ };

\draw[dashed,thick] (1.8,-2.2) -- (1.8,1.2) -- (6.2,1.2) -- (6.2,0.8) -- (2.2,0.8) -- (2.2,-2.2) -- (1.8,-2.2);
\draw[dashed,draw=gray] (2.8,-2.2) -- (2.8,0.2) -- (6.2,0.2) -- (6.2,-0.2) -- (3.2,-0.2) -- (3.2,-2.2) -- (2.8,-2.2);

\end{tikzpicture}

\caption{Interpolation in Ecken und Enden des feinen Gitters}
\label{fig:InterpolationEdgeCase}
\end{figure}

\noindent
So kann es an den Außengrenzen der Simulationsdomäne dazu kommen, dass nur drei der vier benötigten groben Nachbarknoten zur Verfügung stehen, wie in der grün markierten Situation dargestellt.
Je nach Implementierung der Kommunikation zwischen den Gittern, kann die gleiche Einschränkung auch in Ecken des feinen Gitters -- hier blau markiert -- dazu kommen, dass eine Interpolation auf Grundlage von nur drei Nachbarn benötigt wird. Entsprechend erhalten wir nach Berechnung der dividierten Differenzen ein Interpolationspolynom auf drei Stützstellen:
\begin{align*}
\sipolarg{x} :&= \sipolarg{-1} \\
&+ \frac{1}{2}(\sipolarg{1}-\sipolarg{-1})(x+1)\\
&+ \frac{1}{8}(\sipolarg{3}-2\sipolarg{1}+\sipolarg{-1})(x+1)(x-1)
\end{align*}
Auch in dieser Situation ist nur der Wert in \(0\) zur Näherung von \(\ipolarg{\star}{x_{g \to f}^f}\) von Interesse:
\[\sipolarg{0} = \frac{3}{8}\sipolarg{-1} + \frac{3}{4}\sipolarg{1} - \frac{1}{8}\sipolarg{3} \numberthis\label{eq:ipol3ord}\]
Passend zur Anzahl der Stützstellen präsentiert sich dieses Verfahren nach erneutem Einsetzen der Taylor-Entwicklung (\ref{eq:sipolTaylorOrder4}) als eine Interpolationsformel dritter Ordnung:
\[\frac{3}{8}\sipolarg{-1} + \frac{3}{4}\sipolarg{1} - \frac{1}{8}\sipolarg{3} \stackrel{(\ref{eq:sipolTaylorOrder4})}{=} \sipolarg{0} + \mathcal{O}(h^3)\]

Trotz Behandlung dieses Sondernfalls werden die höheren Approximationsordnungen gegenüber (\ref{eq:ipol2ord}) weiterhin und unumgänglich durch zusätzliche Stützstellen erkauft, welche bei parallelisierten LBM-Umsetzungen kommuniziert werden müssen. Es gilt also, zwischen Güte der Interpolation und Anzahl sowie Position der Stützstellen abzuwiegen.

\newpage
\subsection{Algorithmus}\label{kap:Algorithmus}

In der zurückliegenden Sektion haben wir, aufbauend auf der Skalierung von Verteilungsfunktionen zwischen Gitterauflösungen, die Restriktion von fein nach grob sowie die Interpolation von grob zu fein nachvollzogen. Die somit erfassten wesentlichen Grundlagen des Verfeinerungsverfahrens gilt es nun in einem \emph{Kopplungsalgorithmus}~\cite[Kap.~3.5]{lagrava12} zusammenzuführen.

\bigskip

Entsprechend (\ref{eq:gridTime}) müssen für jeden groben Zeitschritt \(\delta t_g\) zwei feine Zeitschritte \(\delta t_f\) durchgeführt werden. Die alternierenden Kollisions- und Strömungsschritte der beiden Gitter sind also strikt gekoppelt und werden als eine Schleifeneinheit betrachtet. Als Schleifeninvariante definieren wir dabei die vollständige Bekanntheit aller Verteilungsfunktionen aller Knoten in beiden Gittern.

\begin{figure}[h]
\tikzexternaldisable
\begin{tikzpicture}[
	scale=1.0,
	lhs/.style={left,xshift=-5mm,align=right,text width=7cm},
	rhs/.style={right,xshift=5mm,align=left,text width=7cm},
	coarse/.style={circle,inner sep=1.5mm},
	fine/.style={circle,inner sep=1mm},
	incomplete/.style={draw=gray,very thick,fill=white},
	complete/.style={draw=gray,very thick,fill=green!70!black}
]


\draw[gray,very thick] (-0.4,0) -- (-0.4,-11) {};
\draw[gray,very thick] (0.4,0) -- (0.4,-11) {};

\node[coarse,complete] at (-0.4,0) {};
\node[lhs] at (-0.4,0) {\(f_{g,i}(\G)\) vollständig zu Zeit \(t\)};

\node[fine,complete] at (0.4,0) {};
\node[rhs] at (0.4,0) {\(f_{f,i}(\F)\) vollständig zu Zeit \(t\)};

\node[coarse,incomplete] at (-0.4,-1) {};
\node[lhs] at (-0.4,-1) {Zeitschritt \(t \to t+\delta t_g\) auf \(\G\)};

\node[fine,incomplete] at (0.4,-2) {};
\node[rhs] at (0.4,-2) {Zeitschritt \(t \to t+\delta t_f\) auf \(\F\)};

\node[coarse,incomplete] at (-0.4,-3) {};
\node[lhs] at (-0.4,-3)
{Interpolation von \(\rho_g, u_g, f_{g,i}^\text{neq}\) in \(x_{g \to f}^g\)};

\node[fine,incomplete] at (0.4,-4) {};
\node[rhs] at (0.4,-4)
{Setzen von \(f_{f,i}\) in \(x_{g \to f}^g\) (\ref{eq:expandedDirectG2F})};

\node[fine,incomplete] at (0.4,-5) {};
\node[rhs] at (0.4,-5)
{Interpolation von \(f_{f,i}\) in \(x_{g \to f}^f\) (\ref{eq:expandedInterpolG2F})};

\node[fine,complete] at (0.4,-6) {};
\node[rhs] at (0.4,-6) {\(f_{f,i}(\F)\) vollständig zu Zeit \(t + \delta t_f\)};

\node[fine,incomplete] at (0.4,-7) {};
\node[rhs] at (0.4,-7) {Zeitschritt \(t+\delta t_f \to t+2\delta t_f\) auf \(\F\)};

\node[fine,incomplete] at (0.4,-8) {};
\node[rhs] at (0.4,-8)
{Setzen von \(f_{f,i}\) in \(x_{g \to f}^g\) (\ref{eq:expandedDirectG2F})};

\node[fine,incomplete] at (0.4,-9) {};
\node[rhs] at (0.4,-9)
{Interpolation von \(f_{f,i}\) in \(x_{g \to f}^f\) (\ref{eq:expandedInterpolG2F})};

\node[coarse,incomplete] at (-0.4,-10) {};
\node[lhs] at (-0.4,-10)
{Restriktion von \(f_{g,i}\) in \(x_{f \to g}\) (\ref{eq:restrictedF2G})};

\node[coarse,complete] at (-0.4,-11) {};
\node[lhs] at (-0.4,-11) {\(f_{g,i}(\G)\) vollständig zu Zeit \(t+\delta t_g\)};

\node[fine,complete] at (0.4,-11) {};
\node[rhs] at (0.4,-11) {\(f_{f,i}(\F)\) vollständig zu Zeit \(t+\delta t_g\)};

\end{tikzpicture}
\tikzexternalenable

\caption{Verfeinerungsalgorithmus mit Invariante aus der Vogelperspektive}
\label{fig:AlgorithmBirdsEye}
\end{figure}
\noindent
Aufbauend auf dieser Invariante ergibt sich die, in Abbildung~\ref{fig:AlgorithmBirdsEye} dargelegte, Reihenfolge der erforderlichen Schritte direkt aus den, für die einzelnen Komponenten der Gitterkopplung benötigen, Informationen. So sind zu Beginn alle Verteilungsfunktionen vollständig bekannt, was die Ausführung eines üblichen Kollisions- und Strömungsschritts (vgl. Kapitel~\ref{kap:LBMimpl}) in beiden Gittern ohne weitere Zuarbeit erlaubt. Nach diesen beiden Schritten fehlen Verteilungsfunktionen \(f_{g,i}(x_{f \to g})\) zur Wiederherstellung der Invariante des groben Gitters. Auch der benötigte zweite Simulationsschritt, um \(\F\) auf Zeitpunkt \(t+\delta t_g=t+2\delta t_f\) zu bringen, scheitert zunächst an der Unbestimmtheit von Verteilungsfunktionen \(f_{f,i}(x_{g \to f})\).

\begin{figure}[h]
\centering
\begin{tikzpicture}[
	scale=0.6,
	coarse/.style={circle,draw=gray,inner sep=2},
	fine/.style={cross out,draw=gray,inner sep=1},
	fchange/.style={very thick,draw=black},
	cchange/.style={very thick,draw=black}
]

\fill [gray!20!white] (12+2.3,1.3) rectangle (12+3.7,5.7);
\fill [gray!20!white] (12+2.3,1.3) rectangle (12+7.3,2.7);
\fill [gray!20!white] (12+2.3,4.3) rectangle (12+7.3,5.7);

\fill [gray!20!white] (1.7,0.7) rectangle (2.3,6.3);
\fill [gray!20!white] (1.7,0.7) rectangle (7.3,1.3);
\fill [gray!20!white] (1.7,5.7) rectangle (7.3,6.3);

\foreach \x in {0,...,3}
	\foreach \y in {0,...,7}
		\node[coarse] at (\x,\y){ };
\foreach \x in {4,...,7}
	\foreach \y in {0,...,2}
		\node[coarse] at (\x,\y){ };
\foreach \x in {4,...,7}
	\foreach \y in {5,...,7}
		\node[coarse] at (\x,\y){ };

\foreach \x in {0,...,10}
	\foreach \y in {0,...,10}
		\node[fine] at (2+0.5*\x,1+0.5*\y){ };

\foreach \x in {0,...,10}
	\node[fine,fchange] at (2+0.5*\x,1){ };
\foreach \x in {0,...,10}
	\node[fine,fchange] at (2+0.5*\x,6){ };
\foreach \y in {1,...,9}
	\node[fine,fchange] at (2,1+0.5*\y){ };

\foreach \x in {0,...,3}
	\foreach \y in {0,...,7}
		\node[coarse] at (12+\x,\y){ };
\foreach \x in {4,...,7}
	\foreach \y in {0,...,2}
		\node[coarse] at (12+\x,\y){ };
\foreach \x in {4,...,7}
	\foreach \y in {5,...,7}
		\node[coarse] at (12+\x,\y){ };

\foreach \x in {0,...,10}
	\foreach \y in {0,...,10}
		\node[fine] at (12+2+0.5*\x,1+0.5*\y){ };

\foreach \x in {0,...,4}
	\node[coarse,cchange] at (12+3+\x,2){ };
\foreach \x in {0,...,4}
	\node[coarse,cchange] at (12+3+\x,5){ };
\foreach \y in {0,...,1}
	\node[coarse,cchange] at (12+3,3+\y){ };

\node at (3.5,8.5) {\(\U_{g \to f}\) zu Zeit \(t+\delta t_f\) und \(t+\delta t_g\):};
\node at (12+3.5,8.5) {\(\U_{f \to g}\) zu Zeit \(t+\delta t_g\):};
\end{tikzpicture}

\caption{Übersicht der zu vervollständigenden Knoten}
\end{figure}

\begin{description}[style=unboxed,leftmargin=0cm]
\item[Vervollständigung von \(\F\) zu Zeitpunkt \(t+\delta t_f\):] Zur Vervollständigung des feinen Gitters nach dem ersten Zeitschritt müssen die fehlenden Verteilungen aus dem groben Gitter rekonstruiert werden. Um die dazu erarbeiteten Kopplungen (\ref{eq:expandedDirectG2F}) und (\ref{eq:expandedInterpolG2F}) anzuwenden, fehlen jedoch Werte der groben Stützstellen \(\smash{f_{g,i}(x_{g \to f}^g})\) zu Zeitpunkt \(t+\delta t_f\). Diese sind zwar in den gesuchten Punkten, dank Trennung der Kopplungsrichtungen durch den Übergangsbereich, nach jedem Simulationsschritt direkt vollständig vorhanden -- jedoch nur zu Zeit \(t\) und \(t+\delta t_g\). Hier findet sich eine Anwendung des Interpolationsverfahrens zweiter Ordnung (\ref{eq:ipol2ord}) zur linearen Zeitinterpolation der benötigten Werte von \(\rho_g, u_g\) und \(f_{g,i}^\text{neq}\):
\[\star(x,t+\delta t_f) \approx \frac{\star(x,t+\delta t_g) + \star(x,t)}{2} \text{ für } \star \in \{\rho_g,u_g,f_{g,i}^\text{neq}\}, x \in \G\]
Aufbauend darauf steht der Anwendung der Kopplungsformeln (\ref{eq:expandedDirectG2F}) und (\ref{eq:expandedInterpolG2F}) zur Vervollständigung von \(\F\) nichts mehr im Wege:
\begin{align*}
f_{f,i}(x_{g \to f}^g,t+\delta t_f) &= f_i^\text{eq}(\rho_g(x_{g \to f}^g,t+\delta t_f), u_g(x_{g \to f}^g,t+\delta t_f)) + \alpha f_{g,i}^\text{neq}(x_{g \to f}^g,t+\delta t_f) \\
f_{f,i}(x_{g \to f}^f,t+\delta t_f) &= f_i^\text{eq}(\ipolarg{\rho_g}{x_{g \to f}^f}, \ipolarg{u_g}{x_{g \to f}^f}) + \alpha \ipolarg{f_{g,i}^\text{neq}}{x_{g \to g}^f}
\end{align*}
Der Interpolationsoperator vierter Ordnung (\ref{eq:ipol4ord}) löst sich dabei für die Zielfunktionen \(\star \in \{\rho_g,u_g,f_{g,i}^\text{neq}\}\) und die Übergangsparallele \(v\) wie folgt auf:
\begin{align*}
\ipolarg{\star}{x_{g \to f}^f} = &\frac{9}{16}(\star(x_{g \to f}^f-\delta x_f v, t+\delta t_f) + \star(x_{g \to f}^f+\delta x_f v, t+\delta t_f))\\
+ &\frac{1}{16}(\star(x_{g \to f}^f-3\delta x_f v, t+\delta t_f) + \star(x_{g \to f}^f+3\delta x_f v, t+\delta t_f))
\end{align*}

\item[Vervollständigung von \(\F\) zu Zeitpunkt \(t+\delta t_g\):] Dieser zweite Rekonstruktionsschritt auf dem feinen Gitter gestaltet sich einfacher, da die benötigten groben Verteilungen in \(\U_{g \to f}\) zur Zeitpunkt \(t+\delta t_g\) bereits durch den initialen Simulationschritt auf dem groben Gitter bekannt sind. Entsprechend können die Kopplungsformeln (\ref{eq:expandedDirectG2F}) und (\ref{eq:expandedInterpolG2F}) direkt zur Vervollständigung von \(\F\) angewandt werden.

\item[Vervollständigung von \(\G\) zu Zeitpunkt \(t+\delta t_g\):] Nach zweimaliger Vervollständigung des feinen Gitters verbleibt zur Wiederherstellung der Schleifeninvariante der Abschluss des eingehenden Kollisions- und Strömungsschritts auf dem groben Gitter durch Restriktion der aus Richtung des feinen Gitters eingehenden Verteilungsfunktionen. Hierzu erlaubt die, durch die zuvorkommenden Schritte garantierte, Vollständigkeit des feinen Gitters zu Zeitpunkt \(t+\delta t_g\), die direkte Anwendung der Kopplungsformel (\ref{eq:restrictedF2G}) mit Restriktionsoperator (\ref{eq:neqAvgRestrictionF2G}) auf die Knoten in \(\U_{f \to g}\).
\begin{align*}
f_{g,i}(x_{f \to g},t+\delta t_g) &= f_i^\text{eq}(\rho_f(x_{f \to g},t+\delta t_g), u_f(x_{f \to g},t+\delta t_g))\\
&+ \frac{1}{\alpha} \frac{1}{q} \sum_{j=0}^{q-1} f_{f,i}^\text{neq}(x_{f \to g} + \delta x_f \xi_j,t+\delta t_g)
\end{align*}
\end{description}
Zu erwähnen bleibt, dass wir aus Konsistenzgründen alle Kopplungsformeln immer auf alle -- und nicht nur die fehlenden -- diskreten Richtungen \(i \in [q-1]\) einer betrachteten Zelle \(x\) anwenden.

\bigskip
Nach Durchführung der drei Vervollständigungsschritte haben wir die Invariante für \(t+\delta t_g\) wieder hergestellt und können von vorne beginnen. Wir haben damit an dieser Stelle das Verfeinerungsverfahren von Lagrava et al. vollständig nachvollzogen und können mit der Implementierung in OpenLB fortfahren.

% ToDo: Randfälle der Restriktion ausarbeiten, analog zu Interpolation (fehlt im Paper)
% ToDo: Experimentelle Begründung, warum Kopplungsformel immer auf alle Richtungen angewandt wird
% ToDo: Bemerkungen zu MPI Unterstützung

\newpage
\section{Implementierung in OpenLB}

OpenLB~\cite{olb12} ist ein umfangreiches frei verfügbares C++ Framework zur Implementierung von LBM basierenden Simulationen. Schwerpunkte sind dabei eine große Flexibilität in Hinblick auf die unterstützten Lattice Boltzmann Modelle sowie weitreichende Modularisierung zur einfachen Umsetzung neuer Anwendungen bei gleichzeitiger Eignung für hochperformante Berechnungen durch Skalierbarkeit auf parallele Großrechner.

\bigskip
Eine LBM Bibliothek mit idealer Unterstützung für Gitterverfeinerung würde es erlauben, dass zu lösende physikalische Problem zunächst ohne Rücksicht auf die konkrete Zusammensetzung des Gitters zu modellieren. Würde dann im Konflikt zwischen Rechenressourcen und angestrebter Ergebnisqualität ein Bedarf für Gitterverfeinerung festgestellt, sollte diese a posteriori ohne weitere Anpassungen hinzugeschalten werden können. In einer solchen Umgebung wäre es dann sogar denkbar, die Entscheidung über Position und Ausmaß der Verfeinerung an ein automatisches Kriterium \cite{lagrava15} zu übertragen.

Diese Vision ist selbstredend von großem Anspruch und weit außerhalb der Grenzen dieser Arbeit. Ein erster Schritt muss jedoch die Ergänzung von OpenLB um grundlegende Unterstützung für manuelle Gitterverfeinerung sein -- d.h. Position, Größe und Auflösung des zu verfeinernden Bereichs müssen durch den Nutzer vorgegeben werden können.

\subsection{Architekturübersicht}

OpenLB verwaltet die Diskretisierung der zu modellierenden Simulationsdomäne in einer \class{CuboidGeometry2D} Instanz. Diese Klasse teilt eine gegebene physikalische Domäne in eine beliebige Anzahl von quaderförmigen \class{Cuboid2D} Instanzen mit definierter Auflösung ein. Diese Aufteilung dient der Parallelisierung, da die Gitter dieser einzelnen Quader außerhalb von kommunizierenden Randbereichen vollständig nebenläufig verarbeitet werden können. Entsprechend werden diese Teilbereiche von einer Implementierung der \class{LoadBalancer} Schnittstelle auf die zu Verfügung stehenden Prozessoren aufgeteilt.

Auf Grundlage dieses räumlichen Umrisses wird dann das eigentliche Gitter in einer \class{SuperLattice2D} Instanz abhängig des gewählten Lattice Boltzmann Modells aufgebaut. Die einzelnen Gitterzellen werden dabei von \class{Cell} Klassen abgebildet, welche anhand des spezifischen, durch einen sogenannten Deskriptor \emph{beschriebenen} und von \class{Dynamics} Instanzen durchgeführten, Kollisionsschrittes der lokalen Umsetzung des LBM Modells Sorge tragen.

Diese \class{Dynamics} beschreiben dabei zusammen mit zellübergreifenden Postprozessoren nicht nur das einfache Strömungsverhalten, sondern modellieren auch die zahlreichen Randkonditionen, welche die Abbildung komplexer Geometrien in LBM erst ermöglichen.
Zur einfacheren Zuordnung dieser zellspezifischen Klassen verwendet OpenLB sogenannte Materialzahlen, welche von einer \class{SuperGeometry2D} Instanz verwaltet werden.

Parallel zu diesen Strukturen kapselt und berechnet die \class{UnitConverter} Klasse etwaig benötigte Konstanten zur Konvertierung zwischen physikalischen Einheiten und den, diese modellierenden, Lattice-Einheiten sowie Relaxionszeiten und Fluidkonstanten wie die Reynolds-Nummer.

\bigskip
Im Allgemeinen ergibt sich aus diesen Komponenten folgende übliche Struktur von OpenLB-basierenden Anwendungen \cite[Kap.~2.1]{olb12userguide}:
\begin{enumerate}
	\item Erstellung des \class{UnitConverter} mit den beabsichtigten Gitterkonstanten
	\item Beschreibung der Simulationsdomäne durch Konstruktion einer \class{CuboidGeometry2D}
	\item Bereitstellung eines \class{LoadBalancer} zur Instantierung einer \class{SuperGeometry2D}
	\item Definition der Materialzahlen in einer \method{prepareGeometry} Methode
	\item Konstruktion der \class{SuperLattice2D} Instanz aus der \class{SuperGeometry2D}
	\item Instantierung der benötigten \class{Dynamics} und etwaigen Randkonditionen
	\item Bindung von \class{Dynamics} und Randkonditionen an die, von \class{SuperLattice2D} verwalteten, \class{Cell} Objekte anhand der Materialzahlen in einer \method{prepareLattice} Methode
	\item Starten der Simulationsschleife zum Aufruf von \method{SuperLattice2D::collideAndStream}
\end{enumerate}

In letzterem, die eigentliche Simulation durchführendem, Schritt, werden weiter durch kanonisch benannte Funktionen wie \method{getResults} und \method{error} die Ergebnisse zur Analyse in Dateien geschrieben, Fehlernormen berechnet und Konvergenzkriterien bestimmt.

\subsection{Auswahl der Verfeinerungsmethode}
\label{sec:olbRefinementChoice}

Ein erster Gedanke zur Integration von Gitterverfeinerung in OpenLB ist die Nutzung der bestehenden Aufteilung der Simulationsdomäne in, durch \class{Cuboid2D} beschriebene, Quader. Insbesondere aus Sicht des Einfügens von Gitterverfeinerung in die bestehende Architektur, sowie der unveränderten Weiterverwendung der \class{LoadBalancer} Algorithmen zur Steuerung der Parallelisierung, scheint ein solcher Ansatz sinnvoll.

Bei Variation der Auflösung einzelner Quader im Rahmen der \class{CuboidGeometry2D} Struktur handelte es sich zwangsweise um einen Multi-Domain Ansatz. Gingen wir diesen Weg, benötigten wir zunächst \class{Cuboid2D} spezifische \class{UnitConverter} Instanzen zur Verwaltung der auflösungsabhängigen Konstanten. Dies müsste dann im Rahmen von \method{prepareLattice} zur Setzung der dann ebenfalls quaderspezifischen \class{Dynamics} und Randkonditionen beachtet werden.

Zur Ermöglichung von Parallelisierung berücksichtigt die, der Gitterverwaltung in \class{SuperLattice2D} zugrundeliegende, Aufteilung der Domäne durch \class{CuboidGeometry2D} bereits Übergangsbereiche, deren Funktion mit zusätzlicher Auflösungskopplung in Einklang zu bringen wäre.

Weiter würde das Problemfeld eben dieser Dekomposition um die Restriktion auf Auflösungsübergänge im Verhältnis \(1:2\) erweitert. So müsste ein guter Algorithmus zur Dekomposition der Simulationsdomäne dann die Anforderungen an Parallelisierung, Rechenauslastung, Geometrie und Verfeinerung sinnvoll auflösen und zugleich manuelle Eingriffe erlauben. Diese zusätzliche starke Einschränkung sowie der dann bei Anpassung der Verfeinerungsstruktur unumgängliche komplette Neuaufbau der Simulation bilden ein schwerwiegendes Gegenargument zu diesem ersten Gedanken.

\bigskip
Der tatsächlich umgesetzte Ansatz ergibt sich aus dem Verständnis von Gitterverfeinerung als Kopplung von ansonsten komplett allein stehenden Simulationen. Die Übergangsbereiche wären in diesem Modell mit Randkonditionen vergleichbar, wie sie für Ein- und Ausflüsse verwendet werden. Gitterverfeinerung könnte so weitestgehend von der bestehenden Architektur getrennt ergänzt werden, was insbesondere auch die veränderungsfreie Unterstützung existierender Anwendungen begünstigen würde.

Eine solche nebenläufige Überlagerung von Simulationen mit jeweils komplett eigenständig verwalteten Gittern gebietet sich bei erster Betrachtung als klarer Umriss eines Multi-Grid Verfahrens. Beachten wir jedoch, dass es einfach möglich ist, die überlagerten Gitterflächen durch \emph{Nullen} der entsprechenden Materialzahlen effizient aus der Verarbeitung auszuschließen und trotzdem bei Bedarf -- z.B. in Hinblick auf Verschiebung von verfeinerten Bereiche während des Simulationsverlaufs -- zu reaktivieren, stellen sich auch Multi-Domain Ansätze in diesem Modell als sinnvoll implementierbar heraus. Vorteil ist hier also gerade auch, dass prinzipiell beide Ansätze zur Gitterverfeinerung umgesetzt werden können und wir nicht durch Festhalten an der existierenden Struktur auf Multi-Domain Verfahren beschränkt sind. Da die Positionierung der Gitter in diesem Ansatz komplett frei wäre, ließen sich aus Architektursicht auch nicht-koinzidierende oder sogar zueinander rotierte Verfeinerungsgitter abbilden.

Ein Vorbild für dieses Konzept zur Umsetzung von Gitterverfeinerung existiert in Form der Optimierungskomponente von OpenLB, welche ebenfalls komplette Simulationen in einem sogenannten \class{Solver} kapselt. Langfristig könnten mit diesem Ansatz also beide gitterübergreifenden Module in einem gemeinsamen Konzept abgebildet werden.

\bigskip
Nachdem nun das grobe Umfeld eines Gitterverfeinerungsframeworks feststeht, gilt es, ein geeignetes Verfahren zur Umsetzung in und Nutzung mit eben diesem Framework zu wählen. Das von Lagrava et al. in \citetitle{lagrava12}~\cite{lagrava12} beschriebene Verfahren, welches insbesondere auf \cite{dupuisChopard03} und \cite{filippova98} einen anpassungsfähigen Multi-Domain Gitterverfeinerungsalgorithmus für BGK LBM auf koinzidierenden D2Q9 Gittern aufbaut, erscheint hier als guter Kandidat. Die anfängliche Beschränkung auf zwei Dimensionen passend zur Einschränkung dieser Arbeit sowie die Flexibilität in Hinblick auf die verwendeten Restriktions- und Interpolationsoperatoren bilden hier eine gute Grundlage für eine erste und doch ausbaufähige Umsetzung von Gitterverfeinerung in OpenLB.

\newpage
\subsection{Struktur des Gitterverfeinerungsframework}

Wie im vorangehenden Kapitel erläutert, soll das Framework zur Gitterverfeinerung nicht tief in vorhandene Strukturen integriert, sondern viel mehr über diesen stehend angesiedelt werden. Eine erste Hürde zu diesem Ziel ist die, größtenteils aus zwangfreien Konventionen bestehende, Struktur von OpenLB Anwendungen. So sind zwar die einzelnen Komponenten der Simulation wie \class{CuboidGeometry2D} und \class{SuperLattice2D} vorgegeben, deren Konstruktion und Verknüpfung erfolgt jedoch größtenteils manuell.

Für sich ist diese Herangehensweise des flexiblen Zusammensetzens von Modulen durchaus erhaltenswert und bildet eine der Stärken von OpenLB. Zur übergreifenden und für den Nutzer möglichst bequemen Einbindung von Gitterverfeinerung -- wir erinnern uns: das Ziel ist es, Gitter erst im Nachhinein mit einem einzigen Funktionsaufruf zu verfeinern -- muss jedoch zumindest die Konstruktion des auflösungseigenen \class{SuperLattice2D} mit dem dazugehörigen Umfeld aus \class{UnitConverter}, \class{LoadBalancer}, \class{CuboidGeometry2D} und \class{SuperGeometry2D} soweit wie möglich gekapselt werden.

\bigskip
Entsprechend besteht das Framework aus zwei Komponenten: Einer \class{Grid2D} Klasse, die in einem Konstruktoraufruf ein \class{SuperLattice2D} zusammen mit dem benötigten Umfeld instanziiert und einer \class{Coupler2D} Klasse zur Abbildung der Übergänge zwischen mehreren \class{Grid2D} Instanzen. Die Gitterklasse stellt dabei eine Methode \method{Grid2D::refine} bereit, die anhand einer Parametrisierung der zu verfeinernden Domäne ein neues \method{Grid2D} konstruiert und über entsprechende \class{Coupler2D} Objekte mit sich selbst verknüpft. Funktionen wie \method{prepareGeometry} und \method{prepareLattice} können in diesem Umfeld dann durch entsprechende \emph{Getter} mit \class{Grid2D} zusammenarbeiten. Arbeiten diese Funktionen bereits auf Grundlage von analytischen Indikatoren, d.h. unabhängig der Auflösung, können sie sogar ohne Anpassung für alle Gitterauflösungen verwendet werden.

\begin{listing}[H]
\begin{minted}{cpp}
template <typename T, template<typename> class DESCRIPTOR>
RefiningGrid2D<T,DESCRIPTOR>::RefiningGrid2D(
	Grid2D<T,DESCRIPTOR>& parentGrid, Vector<T,2> origin, Vector<T,2> extend):
	Grid2D<T,DESCRIPTOR>(
		std::unique_ptr<IndicatorF2D<T>>(new IndicatorCuboid2D<T>(extend, origin)),
		2*parentGrid.getConverter().getResolution(),     // Auflösungsübergang $2:1$
		2*parentGrid.getConverter().getLatticeRelaxationTime() - 0.5, // Siehe $(\ref{eq:gridTauShift})$
		parentGrid.getConverter().getReynoldsNumber()),               // $\text{Re}_g = \text{Re}_f$
	_origin(origin),
	_extend(extend),
	_parentGrid(parentGrid)
{ }
\end{minted}
\caption{Konstruktor der verfeinernden Gitter}
\label{lst:RefiningGrid}
\end{listing}

Die Konstruktion von \class{Grid2D} erfolgt anhand einer indikatorgegebenen Beschreibung der Simulationsdomäne sowie der gewünschten Auflösung zusammen mit der Relaxionszeit und der modellierenden Reynolds-Nummer:
\begin{minted}{cpp}
Grid2D(FunctorPtr<IndicatorF2D<T>>&& domainF, int resolution, T tau, int re);
\end{minted}
Während sich die Realisierung dieser Signatur als einfache Konstruktion der erläuterten OpenLB Struktur erweist, gestaltet sich die Konstruktion der vererbten \class{RefiningGrid2D} Klasse in Listing~\ref{lst:RefiningGrid} interessanter, da hier dann Kraft der Ergebnisse von Kapitel~\ref{kap:Skalierung} die Vorgabe des groben Gitters zusammen mit dem verfeinerungsbedürftigen Teilbereich zur Erstellung eines neuen Gitters ausreicht.

\begin{listing}[H]
\inputminted{cpp}{code/grid2d_collide_and_stream.cpp}
\caption{Rekursiver Kollisions- und Strömungsschritt mit Gitterkopplung}
\label{lst:GridCollideAndStream}
\end{listing}

Wie in Kapitel~\ref{kap:Algorithmus} dargelegt, müssen zur Gitterkopplung nach jedem Kollisions- und Strömungsschritt verschiedene Arbeiten durchführt werden. So ist die Ausführung von Kollisions- und Strömungsschritten auf dem feinen Gitter zusammen mit der jeweiligen Vor- und Nacharbeit strikt an die Nacharbeit von Kollisions- und Strömungsschritten auf dem groben Gitter gebunden.

Wurde das grobe Gitter um einen Zeitschritt weiterentwickelt, muss der Zustand des feinen Gitters ebenfalls um entsprechend zwei feine Zeitschritte weiterentwickelt werden, damit die groben Verteilungsfunktionen vervollständigt werden können. Es liegt somit nahe, die Aufrufe von \method{SuperLattice2D::collideAndStream} in einer Methode \method{Grid2D::collideAndStream} zu kapseln, um auf diese Weise die Aufrufe von \class{Coupler2D} an den korrekten Stellen durchzuführen.

Konkret erhalten wir in Listing~\ref{lst:GridCollideAndStream} bei gesammelter Verwaltung der von \method{Grid2D::refine} erstellten feinen Gitter und den zugehörigen Kopplern eine, der Algorithmenübersicht in Abbildung~\ref{fig:AlgorithmBirdsEye} nicht unähnliche, Implementierung von \method{Grid2D::collideAndStream}.
Zu bemerken ist, dass die Konstellation aus dieser Methode zusammen mit \method{Grid2D::refine} durch Selbstaufruf bereits die freie Schachtelung von Verfeinerungsbereichen erlaubt.

\begin{listing}[H]
\begin{minted}{cpp}
// Initialisiere gröbstes Gitter mit gewünschten Fluidkonstanten
Grid2D<T,DESCRIPTOR> coarseGrid(coarseDomain, resolution, tau, Re);
prepareGeometry(coarseGrid);

// Verfeinere ein Einheitsquadrat beginnend bei $(0.5,0.5) \in \R^2$
auto fineGrid = coarseGrid.refine({0.5, 0.5}, {1.0, 1.0});
prepareGeometry(fineGrid);

// Schließe den feinen Bereich auf dem groben Gitter aus (optional)
auto refinedOverlap = fineGrid.getRefinedOverlap();
coarseGrid->getSuperGeometry().rename(1,0,*refinedOverlap);

// Binde Dynamics und Randkonditionen an die beiden Gitter
prepareLattice(coarseGrid);
prepareLattice(fineGrid);

// Simulationsschleife mit Ausgabe
for (int iT = 0; iT < coarseGrid->getConverter().getLatticeTime(100); ++iT) {
	coarseGrid->collideAndStream();

	getResults(coarseGrid, iT);
	getResults(fineGrid,   iT);
}
\end{minted}
\caption{Beispielhafte Nutzung von \class{Grid2D}}
\label{lst:RefinementUsageExample}
\end{listing}

Wie in diesem leicht vereinfachten Listing zu sehen, kann Gitterverfeinerung mit dem beschriebenen Framework schon erfreulich kompakt formuliert werden. Tatsächlich fehlt zum Etappenziel der einzeilig aktivierbaren manuell positionierten Gitterverfeinerung nur eine weitere Abstraktion von von \method{prepareGeometry} und \method{prepareLattice}.

\bigskip
Zur umfassenden Beschreibung des Gitterverfeinerungsframework fehlt uns jetzt nur noch die Definition der \class{Coupler2D} Objekte. Da diese die Einzelheiten des Verfeinerungsverfahrens umsetzen, werden sie im Rahmen des folgenden Kapitels zur Implementierung des Verfahrens von Lagrava et al. näher beleuchtet werden.

\newpage
\subsection{Umsetzung des Verfahrens von Lagrava et al.}

Grundsätzlich implementiert jede Instanz von \class{Coupler2D} die Kopplung zweier Gitter in einer Richtung entlang einer, durch Ursprung und Ausdehnung charakterisierten, Linie innerhalb der physikalischen Simulationsdomäne. Für die Kopplung einer rechteckigen \class{RefiningGrid2D} Instanz werden von \method{Grid2D::refine} in diesem Fall acht Kopplungsobjekte erzeugt -- vier Seiten mit jeweils zwei Kopplern.

\begin{listing}[H]
\inputminted{cpp}{code/coupler2d.cpp}
\caption{Gemeinsame Struktur beider Kopplungsklassen}
\label{lst:Coupler2D}
\end{listing}

Die im Zuge dieser Arbeit entwickelte Version von \class{Coupler2D} beschränkt sich hierbei auf zu einem Einheitsvektor parallele Gitterübergänge. Sowohl für die Kopplung der mit \class{CuboidGeometry2D} modellierbaren Aufteilungen als auch für das umzusetzende Verfahren ist diese Einschränkung kein Hindernis, da Lagrava et al. ebenfalls nur von horizontalen bzw. vertikalen Gitterübergängen ausgehen.

Da \class{Grid2D} Methoden zur Diskretisierung physikalischer Koordinaten auf das Gitter bereitstellt, besteht das Fundament der beiden benötigten Kopplungsklassen größtenteils nur aus der Bestimmung aller zu setzenden Kopplungsknoten entlang der Übergangslinie. Die abgeleiteten Klassen \class{FineCoupler2D} und \class{CoarseCoupler2D} können auf die Liste dieser Knoten dann mittels \method{getFineLatticeR} und \method{getCoarseLatticeR} zugreifen und so ihre eigene Implementierung auf das Wesentliche beschränken.

\begin{listing}[H]
\inputminted{cpp}{code/fineCoupler2d.cpp}
\caption{Struktur des Kopplers von grob nach fein}
\label{lst:FineCoupler2D}
\end{listing}

Wie in Listing~\ref{lst:FineCoupler2D} zu sehen, benötigt das Setzen der feinen Verteilungsfunktionen in \method{FineCoupler2D::couple} einen Zwischenspeicher der groben Verteilungsmomente und der groben Nicht-Equilibriumsverteilung entlang der Kopplungslinie. Dieser wird in der Methode \method{FineCoupler2D::store} gesetzt und dient in \method{FineCoupler2D::interpolate} der linearen Zeitinterpolation der groben Verteilungsfunktionen zu Zeitpunkt \(t+\delta t_f\) entsprechend der Ausführungen in Kapitel~\ref{kap:Algorithmus}.

\begin{listing}[H]
\inputminted{cpp}{code/fineCoupler2d_couple_extract.cpp}
\caption{Ausschnitt der Methode \method{FineCoupler2D::couple}}
\label{lst:FineCoupler2D_couple_extract}
\end{listing}

Zur Beleuchtung des Herzstücks der feinen Kopplung sehen wir in Listing~\ref{lst:FineCoupler2D_couple_extract} einen Ausschnitt der Kopplungsfunktion für Knoten \(x_{g \to f}^f \in \F\) mit räumlicher Interpolation der benötigten groben Werte. Zusammen mit der zugehörigen Interpolationsfunktion in Listing~\ref{lst:ipol4ord} lässt sich dabei sehen, wie sich das Verfeinerungsvefahren durch Verwenden von OpenLB Modulen wie den \class{lbHelpers} und der \class{Vector} Datenstruktur sehr nah an seiner mathematischen Formulierung umsetzen lässt.

\begin{listing}[H]
\begin{minted}{cpp}
template <typename T, unsigned N>
Vector<T,N> order4interpolation(const std::vector<Vector<T,N>>& data, int y)
{
	return 9./16. * (data[y] + data[y+1]) - 1./16. * (data[y-1] + data[y+2]);
}
\end{minted}
\caption{Templatefunktion der Interpolationsformel (\ref{eq:ipol4ord})}
\label{lst:ipol4ord}
\end{listing}

Der Koppler \class{CoarseCoupler2D} zum Setzen der groben Verteilungsfunktionen gestaltet sich derweil einfacher, da kein Zwischenspeicher benötigt wird und nur eine Restriktionsoperation anzuwenden ist.

\begin{listing}[H]
\inputminted{cpp}{code/computeRestrictedFneq.cpp}
\caption{Umsetzung der Restriktionsoperation (\ref{eq:neqAvgRestrictionF2G})}
\label{lst:CoarseCoupler2D_restriction}
\end{listing}

Hiermit sind die zentralen Bestandteile der Umsetzung des Verfahrens von Lagrava et al. im Kontext des in dieser Arbeit entwickelten Gitterverfeinerungsframework für OpenLB beschrieben. Zum Abschluss verbleibt nun noch die Evaluierung der Qualität eben dieses Verfahrens anhand verschiedener Beispiele.

\newpage
\section{Evaluierung}

Die Auswahl von Beispielen zur Evaluation der Qualität eines Gitterverfeinerungsverfahrens gestaltet sich ohne Detailkenntnis des Versuchsaufbaus eines real zu simulierenden Strömungsproblem zunächst unklarer, als man annehmen könnte. So stellt OpenLB zwar eine gute Auswahl verschiedener Simulationsbeispiele bereit, aber nur wenige von ihnen beinhalten auch analytische Lösungen oder einfach zu verwendende Vergleichsdaten.

Unter diesen Einschränkungen ist es -- abseits offensichtlicher Gütekriterien wie dem Ausschluss divergierender Simulationen -- schwer zu sagen, ob zwei auf verschiedenen Weisen simulierte Lösungen nun besser oder schlechter als die jeweils andere Lösung sind. Entsprechend beschränken wir uns je nach Beispiel auf die Betrachtung einer Auswahl der folgenden Gütekriterien:
\begin{enumerate}
	\item Subjektive Qualität des Strömungsbildes
	\item Erhalt von Masse im Gitterübergang
	\item Stetigkeit der Momente im Gitterübergang
	\item Vergleich der Lösungen von lokal verfeinerten und global hochaufgelösten Gittern
\end{enumerate}

\subsection{Rohrströmung}

Als Einstiegspunkt wollen wir zunächst die grundsätzliche Funktion des Verfeinerungsverfahren an einem möglichst einfachen Beispiel gegen möglichst korrekte Vergleichsdaten testen. Ein solches Beispiel ist gegeben durch die laminare Strömung in einem Rohr mit kreisförmigem Querschnitt und ohne Hindernisse abseits der Wände.

Eine solche Rohrströmung stellt nicht nur eine der einfachsten Strömungssituationen dar, sondern besitzt als Poiseuille-Fluss auch eine analytische Lösung, so dass wir ideale Vergleichsbedingungen vorfinden. Lieferte unser Verfahren in diesem Beispiel keine guten Ergebnisse, wäre nicht davon auszugehen, dass dies sich in komplexeren Situationen verbessern würde.

\begin{Definition}[Analytische Lösung des Poiseuille-Flusses]
\label{def:analyticPoiseuille}
Seien \(L_x, L_y \in \R_+\) die räumlichen Rohrdimensionen, \(\nu\) die kinematische Viskosität und \(\Delta p := p_1 - p_0\) die Druckdifferenz zwischen Ein- und Ausfluss. Dann ist die analytische Geschwindigkeit in \(x\)-Richtung gegeben als \cite[vgl.~Kap.~4]{bao11}:
\[u_x(y) = \frac{1}{2\nu} \frac{\Delta p}{L_x} y (y-L_y)\]
Dies kann mit \(u_x(L_y/2):=1\) und \(\Delta p = 1 - p_0\) vereinfacht werden zu:
\[u_x(y) = -\frac{4}{L_y^2} y (y-L_y)\]
Das Geschwindigkeitsprofil des Poiseuille-Flusses ist also parabelförmig.
\end{Definition}

Wir wollen in einem \(1 \times 4\) Längeneinheiten bemessenden Rohr einen Poiseuille-Fluss simulieren. Als Auflösung einer Längeneinheit sei dabei \(N=10\) gewählt, was in der Gitterdiskretisierung durch \(11 \times 21\) grobe und \(21 \times 43\) feine Knoten abgebildet wird. In Abbilddung~\ref{fig:PoiseuilleGridSetup} sehen wir das resultierende Gitter zusammen mit den zugewiesenen Materialzahlen.

\begin{figure}[h]
\begin{adjustbox}{center}
\begin{tikzpicture}
\begin{axis}[
	scale only axis,
	height=\textwidth)/4,
	width=\textwidth,
	axis equal,
	xmin=0, xmax=4,
	ymin=0, ymax=1,
	x tick label style={/pgf/number format/.cd, use comma},
	y tick label style={/pgf/number format/.cd, use comma},
	scatter/classes={
		1={color=blue!50!white},
		2={color=gray!70!black},
		3={color=green!70!black},
		4={color=red!70!black}
	},
	legend style={
		at={(0.5,-0.2)},
		anchor=north,
		draw=none,
		column sep=1ex
	},
	legend columns=4,
	legend image post style={scale=1.5}
]

\addplot[scatter,only marks,point meta=explicit] table[
	x=x, y=y, point meta=\thisrow{geometry},
	col sep=semicolon,
	/pgf/number format/read comma as period
] {img/data/poiseuille2d_re10_n10_grid.csv};

\addlegendentry{Fluid}
\addlegendentry{Wand}
\addlegendentry{Einfluss}
\addlegendentry{Ausfluss}
\end{axis}
\end{tikzpicture}

\end{adjustbox}
\caption{Gitterstruktur einer halbseitig verfeinerten Rohrströmung}
\label{fig:PoiseuilleGridSetup}
\end{figure}

Wand- und Einflusszellen werden nach dieser Vorlage mit Geschwindigkeitsrandbedingungen umgesetzt. Während für den Einfluss dabei das Geschwindigkeitsprofil der analytischen Poiseuille-Lösung vorausgesetzt wird, erhält der Ausfluss eine Druckrandbedingung. Die noch verbleibenden normalen Fluidzellen erfahren, abgesehen von den üblichen Kollisions- und Strömungsschritten, keine besondere Behandlung.

\begin{figure}[h]
\begin{adjustbox}{center}
\begin{tikzpicture}
\begin{axis}[
	scale only axis,
	height=\textwidth)/4,
	width=\textwidth,
	axis equal,
	xmin=0, xmax=4,
	ymin=0, ymax=1,
	x tick label style={/pgf/number format/.cd, use comma},
	y tick label style={/pgf/number format/.cd, use comma},
	point meta=explicit,
	colorbar,
	colormap/RdYlBu-4,
	colormap={reverse RdYlBu-4}{
		indices of colormap={
			\pgfplotscolormaplastindexof{RdYlBu-4},...,0 of RdYlBu-4
		}
	},
	colorbar horizontal,
	colorbar style={
		/pgf/number format/precision=2,
		/pgf/number format/use comma,
		at={(0.5,-0.2)},
		anchor=north,
		width=0.5*\pgfkeysvalueof{/pgfplots/parent axis width},
		y unit=m/s,
		y label style={
			at={(axis description cs:1.3,0.5)},
			rotate=-90,
			anchor=east
		}
	},
	scatter/use mapped color={color=mapped color}
]

\addplot[scatter,only marks] table[
	x=x, y=y, meta=ux,
	col sep=comma
] {img/data/poiseuille2d_refined_re10_n10_grid.csv};
\end{axis}
\end{tikzpicture}

\end{adjustbox}
\caption{Geschwindigkeiten in \(x\)-Richtung im simulierten Poiseuille-Fluss}
\label{fig:PoiseuilleVelocityGrid}
\end{figure}

Neben diesen knotenspezifischen Eigenschaften sei \(u=0.01\) die charateristische Geschwindigkeit in Lattice-Einheiten und \(\text{Re}=10\) die modellierte Reynolds-Zahl. Erstellen wir unsere grobe \class{Grid2D} Instanz mit diesen, die Relaxionszeit \(\tau\) fixierenden, Werten und führen die Simulation bis zur Konvergenz aus, erblicken wir bei geeigneter Aufbereitung in ParaView~\cite{paraview05} schließlich das in Abbildung~\ref{fig:PoiseuilleVelocityGrid} ersichtliche Strömungsbild. Konvergenz bedeutet in diesem Kontext, dass die durchschnittliche Energie des feinen Gitters unter einen Residuumswert, hier \(1\mathrm{e}{-5}\), gefallen ist.

Bei erster Betrachtung lässt sich erkennen, dass die Strömung den Gitterübergang subjektiv ideal bestritten hat. Es treten also keine ungewöhnlichen Artefakte im Geschwindigkeitsbild auf und dieses setzt sich nach dem Übergang in, bis auf die neuen Zwischenwerte, unveränderter Weise fort. Tatsächlich ist bei Interpolation der Knotenzwischenbereiche zur Bildung einer geschlossenen Fläche kein Gitterübergang erkennbar.

\bigskip

Zur formalen Qualitätsbewertung ziehen wir im Folgenden die analytische Lösung~\ref{def:analyticPoiseuille} von Geschwindigkeit und Druck des Poiseuille-Flusses heran. Wir können diese in \mbox{OpenLB} einfach mit Hilfe des \class{SuperRelativeErrorL2Norm2D} Funktors auf beiden Gittern mit der simulierten Lösung vergleichen:

\begin{center}
\begin{tabular}{ l l l l }
Gitterstruktur & Geschwindigkeitsfehler & Druckfehler \\
\hline
\(11 \times 41\) & 1.88337e-3 & 3.11534e-3 \\
\hline
\(21 \times 81\) & 1.16253e-3 & 2.27404e-3 \\
\hline
\(11 \times 21\) & 1.00761e-2 & 2.71447e-2 \\
\hphantom{\(11 \times 21\)} \(21 \times 43\) & 1.02424e-2 & 2.07924e-2 \\
\hline
\(11 \times 41\) & 1.77561e-3 & 3.11244e-3 \\
\hphantom{\(11 \times 41\)} \(17 \times 41\) & 2.04335e-3 & 3.39432e-3 \\
\end{tabular}
\end{center}

\noindent
Die halbseitig verfeinerte Lösung aus Abbildung~\ref{fig:PoiseuilleGridSetup} ist zunächst also um eine Größenordnung schlechter als eine gleichmäßig mit \(n=20\) aufgelöste Berechnung. Etwas überraschend ist ihr Fehler auch deutlich größer als der Fehler einer uniform mit \(n=10\) aufgelösten Simulation -- zumindest in diesem speziellen Beispiel hat Gitterverfeinerung also einen messbar negativen Einfluss auf die Genauigkeit der Simulation.

Nicht vergessen werden sollte jedoch, dass die untersuchte halbseitig verfeinerte Poiseuille-Strömung als Beispiel sehr konstruiert und nicht realitätsnah ist. Auch die noch folgenden Beobachtungen in Abbildung~\ref{fig:PoiseuilleOutflowProfile}, nach welchen die lineare Interpolation zu kleineren Geschwindigkeitsfehlern führt, deutet auf eine beschränkte Aussagekraft dieses Beispiels hin.
Eine Besonderheit ist in diesem Kontext auch die Existenz von Randbedingungen im Übergangsbereich. Eine etwaige Behandlung dieser wurde weder von Lagrava et al. angesprochen, noch in der dem Leser vorliegenden Arbeit näher untersucht. Tatsächlich verschwindet der \emph{Verfeinerungsfehler} fast vollständig, wenn die Wände aus dem verfeinerten Bereich ausgespart werden.

\bigskip

Abschließend erscheint es beispielübergreifend intuitiv erwartbar, dass eine nicht aus dem konkreten Strömungsproblem informierte Anwendung von Gitterverfeinerung -- und damit eine unbegründete Erhöhung der Simulationskomplexität gegenüber einem uniformen Gitter -- einer Verbesserung der Präzision nicht zuträglich ist.

\newpage
\subsubsection{Vergleich der Interpolationsverfahren}

Den Poiseuille Simulationsaufbau können wir an dieser Stelle auch zur Nachvollziehung des, von Lagrava et al. für die Verwendung eines Verfahrens vierter Ordnung in der räumlichen Interpolation feiner Übergangsknoten ohne koinzidierende grobe Stützpunkte hervorgebrachten, Arguments verwenden.

\begin{figure}[h]
\begin{adjustbox}{center}
\begin{tikzpicture}
\begin{axis}[
	scale only axis,
	height=8.75cm,
	width=0.9*\textwidth,
	mark size=0,
	line width=0.5pt,
	legend pos=south west,
	xlabel=Horizontale Kanalposition (\(m\)),
	ylabel=Druck (\(N/m^2\))
]

\addplot table[
	x=x, y=cubic,
	col sep=semicolon,
	/pgf/number format/read comma as period
]{img/data/poiseuille2d_bisected_re100_both.csv};
\addlegendentry{Druckverlauf bei Interpolationsordnung \(\mathcal{O}(h^4)\)};

\addplot table[
	x=x, y=linear,
	col sep=semicolon,
	/pgf/number format/read comma as period
]{img/data/poiseuille2d_bisected_re100_both.csv};
\addlegendentry{Druckverlauf bei Interpolationsordnung \(\mathcal{O}(h^2)\)};

\draw [dotted] (axis cs:1.9,1.35) rectangle (axis cs:2.09,1.45);
\end{axis}

\begin{axis}[
	yshift=4.5cm,
	xshift=9cm,
	scale only axis,
	height=3cm,
	width=4cm,
	mark size=0,
	line width=0.5pt,
	ticks=none,
	title=Übergangsbereich
]

\addplot table[
	x=x, y=cubic,
	col sep=semicolon,
	/pgf/number format/read comma as period
]{img/data/poiseuille2d_bisected_re100_detail.csv};

\addplot table[
	x=x, y=linear,
	col sep=semicolon,
	/pgf/number format/read comma as period
]{img/data/poiseuille2d_bisected_re100_detail.csv};
\draw [dashed,gray] (axis cs:1.98,\pgfkeysvalueof{/pgfplots/ymin}) -- (axis cs:1.98,\pgfkeysvalueof{/pgfplots/ymax});
\draw [dashed,gray] (axis cs:2.00,\pgfkeysvalueof{/pgfplots/ymin}) -- (axis cs:2.00,\pgfkeysvalueof{/pgfplots/ymax});
\end{axis}
\end{tikzpicture}

\end{adjustbox}
\caption{Druckverlauf bei linearer und kubischer Interpolation \cite[vgl.~Abb.~11]{lagrava12}}
\label{fig:PoiseuilleMassloss}
\end{figure}

Wir setzen dazu \(N=50\) als Auflösung der Längeneinheit, \(\text{Re}=100\) als Reynolds-Zahl und eine Geschwindigkeitsrandbedingung mit Poiseuilleprofil im Ausfluss. Führen wir dann die Simulation mit dem linearen Interpolationsverfahren (\ref{eq:ipol2ord}) aus und vergleichen den Verlauf des physikalischen Drucks auf einer vertikal zentrierten horizontalen Linie mit den, aus einem Durchlauf mit dem Verfahren vierter Ordnung (\ref{eq:ipol4ord}) gewonnen, Daten, erhalten wir den in Abbildung~\ref{fig:PoiseuilleMassloss} zu sehenden Plot.

Entsprechend der Beobachtungen in \cite[Kap.~3.7]{lagrava12} sehen auch wir bei linearer Interpolation einen prominenten Abfall des physikalischen Drucks im Übergangsbereich der Gitter. Bei kubischer Interpolation tritt dieser Fehler nicht auf, der Druckverlauf ist in diesem Fall so glatt, dass der Übergang nicht mehr zu erkennen ist.

\begin{figure}[h]
\begin{adjustbox}{center}
\begin{tikzpicture}
\pgfplotstableread[col sep=comma]{img/data/poiseuille2d_refined_order4ipol_velocity_profile_at_3.csv}\refinedCubicIpol
\pgfplotstableread[col sep=comma]{img/data/poiseuille2d_refined_order2ipol_velocity_profile_at_3.csv}\refinedLinearIpol
%\pgfplotstableread[col sep=comma]{img/data/poiseuille2d_unrefined_coarse_velocity_profile_at_3.csv}\unrefinedCoarse
\pgfplotstableread[col sep=comma]{img/data/poiseuille2d_unrefined_fine_velocity_profile_at_3.csv}\unrefinedFine

\begin{axis}[
	scale only axis,
	height=8cm,
	width=0.4*\textwidth,
	mark size=4,
	legend cell align=left,
	legend style={at={(0.9,-0.1)},anchor=north},
	grid=both,
	ylabel=\(x\)-Geschwindigkeit
	xtick={0,0.25,0.5,0.75,1},
	title=Geschwindigkeitsprofil,
	every axis plot/.append style={thick}
]

\addplot[
	only marks,
	mark=x,
	color=green!70!black
] table [
	x=y, y=ux
] {\refinedCubicIpol};
\addlegendentry{Halbseitig verfeinertes Gitter mit kubischer Interpolation};

\addplot[
	only marks,
	mark=+,
	color=red!70!black
] table [
	x=y, y=ux
] {\refinedLinearIpol};
\addlegendentry{Halbseitig verfeinertes Gitter mit linearer Interpolation};

\addplot[
	only marks,
	mark=asterisk,
	color=blue!50!white
] table[
	x=y, y=ux
] {\unrefinedFine};
\addlegendentry{Uniform fein aufgelöstes Gitter};

\addplot [domain=0:1, samples=100]{-4*x*(x-1)};
\addlegendentry{Analytische Lösung}
\end{axis}

\begin{axis}[
	scale only axis,
	height=8cm,
	width=0.4*\textwidth,
	mark size=4,
	grid=both,
	xtick={0,0.25,0.5,0.75,1},
	xshift=7cm,
	scaled y ticks=false,
	ylabel=Fehler,
	yticklabel pos=right,
	y tick label style={/pgf/number format/sci},
	title=Fehler zur analytischen Lösung,
	every axis plot/.append style={thick}
]

\addplot[
	only marks,
	mark=x,
	color=green!70!black
] table [
	x=y,
	y expr=\thisrow{ux}+4*\thisrow{y}*(\thisrow{y}-1)
] {\refinedCubicIpol};

\addplot[
	only marks,
	mark=+,
	color=red!70!black
] table [
	x=y,
	y expr=\thisrow{ux}+4*\thisrow{y}*(\thisrow{y}-1)
] {\refinedLinearIpol};

\addplot[
	only marks,
	mark=asterisk,
	color=blue!50!white
] table[
	x=y,
	y expr=\thisrow{ux}+4*\thisrow{y}*(\thisrow{y}-1)
] {\unrefinedFine};

\addplot [domain=0:1, samples=100]{0};
\end{axis}
\end{tikzpicture}

\end{adjustbox}
\caption{Vergleich des vertikalen Geschwindigkeitsprofil bei \(x=3\)}
\label{fig:PoiseuilleOutflowProfile}
\end{figure}

\newpage
\subsection{Umströmter Zylinder}

\begin{figure}[h]
\begin{adjustbox}{center}
\includegraphics[width=1.2\textwidth]{img/static/cylinder2d_unrefined_60s_full.pdf}
\end{adjustbox}
\caption{Uniform aufgelöstes Strömungsbild zu \(t=60s\)}
\end{figure}

\begin{figure}[h]
\begin{adjustbox}{center}
\includegraphics[width=1.2\textwidth]{img/static/cylinder2d_single_refinement_60s_full.pdf}
\end{adjustbox}
\caption{Einfach verfeinertes Strömungsbild zu \(t=60s\)}
\end{figure}

\begin{figure}[h]
\begin{adjustbox}{center}
\includegraphics[width=1.2\textwidth]{img/static/cylinder2d_single_refinement_60s_knudsen_full.pdf}
\end{adjustbox}
\caption{Einfach verfeinertes Knudsenkriterium zu \(t=60s\)}
\end{figure}

\subsection{Dipol}

\newpage
\section{Fazit}
